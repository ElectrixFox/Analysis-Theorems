% This file makes a printable version of the blueprint
% It should include all the \usepackage needed for the pdf version.
% The template version assume you want to use a modern TeX compiler
% such as xeLaTeX or luaLaTeX including support for unicode
% and Latin Modern Math font with standard bugfixes applied.
% It also uses expl3 in order to support macros related to the dependency graph.
% It also includes standard AMS packages (and their improved version
% mathtools) as well as support for links with a sober decoration
% (no ugly rectangles around links).
% It is otherwise a very minimal preamble (you should probably at least
% add cleveref and tikz-cd).

\documentclass[a4paper]{article}

\usepackage{geometry}
\ProvidesPackage{preamble}

% General setup
\usepackage{amsmath, amsfonts, amssymb, amsthm}
\usepackage[pdfencoding=unicode, psdextra]{hyperref}
\usepackage{mathtools}
\usepackage{enumitem}
\usepackage{thmtools}
\usepackage{microtype}
\usepackage{pgfplots}
\usepackage{parskip}
\usepackage{tikz}
\usepackage{float}
\pgfplotsset{compat=1.18}

\DeclareMathOperator{\sech}{sech}
\DeclareMathOperator{\csch}{csch}
\newcommand{\N}{\mathbb{N}}
\newcommand{\Z}{\mathbb{Z}}
\newcommand{\Q}{\mathbb{Q}}
\newcommand{\R}{\mathbb{R}}
\newcommand{\C}{\mathbb{C}}

% Analysis I - setup
\newcommand{\infsum}[1][k = 1]{\sum_{#1}^{\infty}}
\newcommand{\infsumo}{\infsum[k = 0]}

% Calculus I - setup
\DeclareMathOperator{\Dom}{Dom}
\DeclareMathOperator{\Ran}{Ran}

\newcommand{\liminfty}[1][n]{\lim_{#1 \rightarrow \infty}}
\newcommand{\pd}[2][]{\frac{\partial #1}{\partial #2}}

% Linear Algebra I - setup
\setcounter{MaxMatrixCols}{20}
\newcommand{\mbf}[1]{\mathbf{#1}}
\DeclareMathOperator{\Adj}{Adj}
\DeclareMathOperator{\im}{im}

\newenvironment{amatrix}[1]
    {
    \left(\begin{array}{@{}*{#1}{c}|c@{}}
        }{
    \end{array}\right)
    }
    
% Probability I - setup

% operators
\DeclareMathOperator{\Bin}{Bin}
\DeclareMathOperator{\Geo}{Geo}
\DeclareMathOperator{\Exp}{Exp}
\DeclareMathOperator{\Po}{Po}

% shortcuts
\renewcommand{\P}{\mathbb{P}}
\newcommand{\E}{\mathbb{E}}
\newcommand{\Var}{\mathbb{V}\mathrm{ar}}
\newcommand{\Cov}{\mathbb{C}\mathrm{ov}}
\newcommand{\ind}[1][A]{\mathbbm{1}_{#1}}
\newcommand{\mmid}{\,\middle|\,}
\newcommand{\cP}[2]{\P\left(#1 \mmid #2\right)}

\declaretheoremstyle[
    notefont = \normalfont\itshape,
    bodyfont = \normalfont\itshape,
    notebraces={(}{)}
]{avgstyle}
\declaretheoremstyle[
    bodyfont = \normalfont,
    notefont = \normalfont\itshape,
    spaceabove = 1em,
    spacebelow = 1em
]{defstyle}
\declaretheoremstyle[
    bodyfont = \normalfont,
    notefont = \normalfont,
    spaceabove = 1em,
    spacebelow = 1em
]{exampstyle}

\declaretheorem[style = avgstyle, numberwithin=subsection]{theorem}
\declaretheorem[numberwithin=subsection, sibling=theorem]{lemma}
\declaretheorem{proposition, corollary, remark}[style=plain, numberwithin=subsection, sibling=theorem]
\declaretheorem{definition}[style=defstyle, numbered=unless unique, numberwithin=subsection, sibling=theorem]
\declaretheorem{outline}[style=plain, numbered=no]
\declaretheorem{example}[style=exampstyle, numberwithin=subsection, sibling=theorem]

\declaretheorem[style = avgstyle, name = Problem]{problem}

\newenvironment{solution}
    {
    \begin{proof}[Solution]\renewcommand{\qedsymbol}{$\triangle$}
    }
    {
    \end{proof}
    }

% for some reason adding a new environment demands I have a new line after it otherwise it breaks.
\usepackage{expl3}

\usepackage{amssymb, amsthm, mathtools}
\usepackage[unicode,colorlinks=true,linkcolor=blue,urlcolor=magenta, citecolor=blue]{hyperref}

\usepackage[warnings-off={mathtools-colon,mathtools-overbracket}]{unicode-math}

% In this file you should put all LaTeX macros and settings to be used both by
% the pdf version and the web version.
% This should be most of your macros.
\DeclareMathOperator{\sech}{sech}
\DeclareMathOperator{\csch}{csch}
\newcommand{\N}{\mathbb{N}}
\newcommand{\Z}{\mathbb{Z}}
\newcommand{\Q}{\mathbb{Q}}
\newcommand{\R}{\mathbb{R}}
\newcommand{\C}{\mathbb{C}}

% Analysis I - setup
\newcommand{\infsum}[1][k = 1]{\sum_{#1}^{\infty}}
\newcommand{\infsumo}{\infsum[k = 0]}

% The theorem-like environments defined below are those that appear by default
% in the dependency graph. See the README of leanblueprint if you need help to
% customize this. 
% The configuration below use the theorem counter for all those environments
% (this is what the [theorem] arguments mean) and never resets it.
% If you want for instance to number them within chapters then you can add
% [chapter] at the end of the next line.
% This file makes a printable version of the blueprint
% It should include all the \usepackage needed for the pdf version.
% The template version assume you want to use a modern TeX compiler
% such as xeLaTeX or luaLaTeX including support for unicode
% and Latin Modern Math font with standard bugfixes applied.
% It also uses expl3 in order to support macros related to the dependency graph.
% It also includes standard AMS packages (and their improved version
% mathtools) as well as support for links with a sober decoration
% (no ugly rectangles around links).
% It is otherwise a very minimal preamble (you should probably at least
% add cleveref and tikz-cd).

\documentclass[a4paper]{article}

\usepackage{geometry}
\ProvidesPackage{preamble}

% General setup
\usepackage{amsmath, amsfonts, amssymb, amsthm}
\usepackage[pdfencoding=unicode, psdextra]{hyperref}
\usepackage{mathtools}
\usepackage{enumitem}
\usepackage{thmtools}
\usepackage{microtype}
\usepackage{pgfplots}
\usepackage{parskip}
\usepackage{tikz}
\usepackage{float}
\pgfplotsset{compat=1.18}

\DeclareMathOperator{\sech}{sech}
\DeclareMathOperator{\csch}{csch}
\newcommand{\N}{\mathbb{N}}
\newcommand{\Z}{\mathbb{Z}}
\newcommand{\Q}{\mathbb{Q}}
\newcommand{\R}{\mathbb{R}}
\newcommand{\C}{\mathbb{C}}

% Analysis I - setup
\newcommand{\infsum}[1][k = 1]{\sum_{#1}^{\infty}}
\newcommand{\infsumo}{\infsum[k = 0]}

% Calculus I - setup
\DeclareMathOperator{\Dom}{Dom}
\DeclareMathOperator{\Ran}{Ran}

\newcommand{\liminfty}[1][n]{\lim_{#1 \rightarrow \infty}}
\newcommand{\pd}[2][]{\frac{\partial #1}{\partial #2}}

% Linear Algebra I - setup
\setcounter{MaxMatrixCols}{20}
\newcommand{\mbf}[1]{\mathbf{#1}}
\DeclareMathOperator{\Adj}{Adj}
\DeclareMathOperator{\im}{im}

\newenvironment{amatrix}[1]
    {
    \left(\begin{array}{@{}*{#1}{c}|c@{}}
        }{
    \end{array}\right)
    }
    
% Probability I - setup

% operators
\DeclareMathOperator{\Bin}{Bin}
\DeclareMathOperator{\Geo}{Geo}
\DeclareMathOperator{\Exp}{Exp}
\DeclareMathOperator{\Po}{Po}

% shortcuts
\renewcommand{\P}{\mathbb{P}}
\newcommand{\E}{\mathbb{E}}
\newcommand{\Var}{\mathbb{V}\mathrm{ar}}
\newcommand{\Cov}{\mathbb{C}\mathrm{ov}}
\newcommand{\ind}[1][A]{\mathbbm{1}_{#1}}
\newcommand{\mmid}{\,\middle|\,}
\newcommand{\cP}[2]{\P\left(#1 \mmid #2\right)}

\declaretheoremstyle[
    notefont = \normalfont\itshape,
    bodyfont = \normalfont\itshape,
    notebraces={(}{)}
]{avgstyle}
\declaretheoremstyle[
    bodyfont = \normalfont,
    notefont = \normalfont\itshape,
    spaceabove = 1em,
    spacebelow = 1em
]{defstyle}
\declaretheoremstyle[
    bodyfont = \normalfont,
    notefont = \normalfont,
    spaceabove = 1em,
    spacebelow = 1em
]{exampstyle}

\declaretheorem[style = avgstyle, numberwithin=subsection]{theorem}
\declaretheorem[numberwithin=subsection, sibling=theorem]{lemma}
\declaretheorem{proposition, corollary, remark}[style=plain, numberwithin=subsection, sibling=theorem]
\declaretheorem{definition}[style=defstyle, numbered=unless unique, numberwithin=subsection, sibling=theorem]
\declaretheorem{outline}[style=plain, numbered=no]
\declaretheorem{example}[style=exampstyle, numberwithin=subsection, sibling=theorem]

\declaretheorem[style = avgstyle, name = Problem]{problem}

\newenvironment{solution}
    {
    \begin{proof}[Solution]\renewcommand{\qedsymbol}{$\triangle$}
    }
    {
    \end{proof}
    }

% for some reason adding a new environment demands I have a new line after it otherwise it breaks.
\usepackage{expl3}

\usepackage{amssymb, amsthm, mathtools}
\usepackage[unicode,colorlinks=true,linkcolor=blue,urlcolor=magenta, citecolor=blue]{hyperref}

\usepackage[warnings-off={mathtools-colon,mathtools-overbracket}]{unicode-math}

% In this file you should put all LaTeX macros and settings to be used both by
% the pdf version and the web version.
% This should be most of your macros.
\DeclareMathOperator{\sech}{sech}
\DeclareMathOperator{\csch}{csch}
\newcommand{\N}{\mathbb{N}}
\newcommand{\Z}{\mathbb{Z}}
\newcommand{\Q}{\mathbb{Q}}
\newcommand{\R}{\mathbb{R}}
\newcommand{\C}{\mathbb{C}}

% Analysis I - setup
\newcommand{\infsum}[1][k = 1]{\sum_{#1}^{\infty}}
\newcommand{\infsumo}{\infsum[k = 0]}

% The theorem-like environments defined below are those that appear by default
% in the dependency graph. See the README of leanblueprint if you need help to
% customize this. 
% The configuration below use the theorem counter for all those environments
% (this is what the [theorem] arguments mean) and never resets it.
% If you want for instance to number them within chapters then you can add
% [chapter] at the end of the next line.
% This file makes a printable version of the blueprint
% It should include all the \usepackage needed for the pdf version.
% The template version assume you want to use a modern TeX compiler
% such as xeLaTeX or luaLaTeX including support for unicode
% and Latin Modern Math font with standard bugfixes applied.
% It also uses expl3 in order to support macros related to the dependency graph.
% It also includes standard AMS packages (and their improved version
% mathtools) as well as support for links with a sober decoration
% (no ugly rectangles around links).
% It is otherwise a very minimal preamble (you should probably at least
% add cleveref and tikz-cd).

\documentclass[a4paper]{article}

\usepackage{geometry}
\ProvidesPackage{preamble}

% General setup
\usepackage{amsmath, amsfonts, amssymb, amsthm}
\usepackage[pdfencoding=unicode, psdextra]{hyperref}
\usepackage{mathtools}
\usepackage{enumitem}
\usepackage{thmtools}
\usepackage{microtype}
\usepackage{pgfplots}
\usepackage{parskip}
\usepackage{tikz}
\usepackage{float}
\pgfplotsset{compat=1.18}

\DeclareMathOperator{\sech}{sech}
\DeclareMathOperator{\csch}{csch}
\newcommand{\N}{\mathbb{N}}
\newcommand{\Z}{\mathbb{Z}}
\newcommand{\Q}{\mathbb{Q}}
\newcommand{\R}{\mathbb{R}}
\newcommand{\C}{\mathbb{C}}

% Analysis I - setup
\newcommand{\infsum}[1][k = 1]{\sum_{#1}^{\infty}}
\newcommand{\infsumo}{\infsum[k = 0]}

% Calculus I - setup
\DeclareMathOperator{\Dom}{Dom}
\DeclareMathOperator{\Ran}{Ran}

\newcommand{\liminfty}[1][n]{\lim_{#1 \rightarrow \infty}}
\newcommand{\pd}[2][]{\frac{\partial #1}{\partial #2}}

% Linear Algebra I - setup
\setcounter{MaxMatrixCols}{20}
\newcommand{\mbf}[1]{\mathbf{#1}}
\DeclareMathOperator{\Adj}{Adj}
\DeclareMathOperator{\im}{im}

\newenvironment{amatrix}[1]
    {
    \left(\begin{array}{@{}*{#1}{c}|c@{}}
        }{
    \end{array}\right)
    }
    
% Probability I - setup

% operators
\DeclareMathOperator{\Bin}{Bin}
\DeclareMathOperator{\Geo}{Geo}
\DeclareMathOperator{\Exp}{Exp}
\DeclareMathOperator{\Po}{Po}

% shortcuts
\renewcommand{\P}{\mathbb{P}}
\newcommand{\E}{\mathbb{E}}
\newcommand{\Var}{\mathbb{V}\mathrm{ar}}
\newcommand{\Cov}{\mathbb{C}\mathrm{ov}}
\newcommand{\ind}[1][A]{\mathbbm{1}_{#1}}
\newcommand{\mmid}{\,\middle|\,}
\newcommand{\cP}[2]{\P\left(#1 \mmid #2\right)}

\declaretheoremstyle[
    notefont = \normalfont\itshape,
    bodyfont = \normalfont\itshape,
    notebraces={(}{)}
]{avgstyle}
\declaretheoremstyle[
    bodyfont = \normalfont,
    notefont = \normalfont\itshape,
    spaceabove = 1em,
    spacebelow = 1em
]{defstyle}
\declaretheoremstyle[
    bodyfont = \normalfont,
    notefont = \normalfont,
    spaceabove = 1em,
    spacebelow = 1em
]{exampstyle}

\declaretheorem[style = avgstyle, numberwithin=subsection]{theorem}
\declaretheorem[numberwithin=subsection, sibling=theorem]{lemma}
\declaretheorem{proposition, corollary, remark}[style=plain, numberwithin=subsection, sibling=theorem]
\declaretheorem{definition}[style=defstyle, numbered=unless unique, numberwithin=subsection, sibling=theorem]
\declaretheorem{outline}[style=plain, numbered=no]
\declaretheorem{example}[style=exampstyle, numberwithin=subsection, sibling=theorem]

\declaretheorem[style = avgstyle, name = Problem]{problem}

\newenvironment{solution}
    {
    \begin{proof}[Solution]\renewcommand{\qedsymbol}{$\triangle$}
    }
    {
    \end{proof}
    }

% for some reason adding a new environment demands I have a new line after it otherwise it breaks.
\usepackage{expl3}

\usepackage{amssymb, amsthm, mathtools}
\usepackage[unicode,colorlinks=true,linkcolor=blue,urlcolor=magenta, citecolor=blue]{hyperref}

\usepackage[warnings-off={mathtools-colon,mathtools-overbracket}]{unicode-math}

% In this file you should put all LaTeX macros and settings to be used both by
% the pdf version and the web version.
% This should be most of your macros.
\DeclareMathOperator{\sech}{sech}
\DeclareMathOperator{\csch}{csch}
\newcommand{\N}{\mathbb{N}}
\newcommand{\Z}{\mathbb{Z}}
\newcommand{\Q}{\mathbb{Q}}
\newcommand{\R}{\mathbb{R}}
\newcommand{\C}{\mathbb{C}}

% Analysis I - setup
\newcommand{\infsum}[1][k = 1]{\sum_{#1}^{\infty}}
\newcommand{\infsumo}{\infsum[k = 0]}

% The theorem-like environments defined below are those that appear by default
% in the dependency graph. See the README of leanblueprint if you need help to
% customize this. 
% The configuration below use the theorem counter for all those environments
% (this is what the [theorem] arguments mean) and never resets it.
% If you want for instance to number them within chapters then you can add
% [chapter] at the end of the next line.
% This file makes a printable version of the blueprint
% It should include all the \usepackage needed for the pdf version.
% The template version assume you want to use a modern TeX compiler
% such as xeLaTeX or luaLaTeX including support for unicode
% and Latin Modern Math font with standard bugfixes applied.
% It also uses expl3 in order to support macros related to the dependency graph.
% It also includes standard AMS packages (and their improved version
% mathtools) as well as support for links with a sober decoration
% (no ugly rectangles around links).
% It is otherwise a very minimal preamble (you should probably at least
% add cleveref and tikz-cd).

\documentclass[a4paper]{article}

\usepackage{geometry}
\ProvidesPackage{preamble}

% General setup
\usepackage{amsmath, amsfonts, amssymb, amsthm}
\usepackage[pdfencoding=unicode, psdextra]{hyperref}
\usepackage{mathtools}
\usepackage{enumitem}
\usepackage{thmtools}
\usepackage{microtype}
\usepackage{pgfplots}
\usepackage{parskip}
\usepackage{tikz}
\usepackage{float}
\pgfplotsset{compat=1.18}

\DeclareMathOperator{\sech}{sech}
\DeclareMathOperator{\csch}{csch}
\newcommand{\N}{\mathbb{N}}
\newcommand{\Z}{\mathbb{Z}}
\newcommand{\Q}{\mathbb{Q}}
\newcommand{\R}{\mathbb{R}}
\newcommand{\C}{\mathbb{C}}

% Analysis I - setup
\newcommand{\infsum}[1][k = 1]{\sum_{#1}^{\infty}}
\newcommand{\infsumo}{\infsum[k = 0]}

% Calculus I - setup
\DeclareMathOperator{\Dom}{Dom}
\DeclareMathOperator{\Ran}{Ran}

\newcommand{\liminfty}[1][n]{\lim_{#1 \rightarrow \infty}}
\newcommand{\pd}[2][]{\frac{\partial #1}{\partial #2}}

% Linear Algebra I - setup
\setcounter{MaxMatrixCols}{20}
\newcommand{\mbf}[1]{\mathbf{#1}}
\DeclareMathOperator{\Adj}{Adj}
\DeclareMathOperator{\im}{im}

\newenvironment{amatrix}[1]
    {
    \left(\begin{array}{@{}*{#1}{c}|c@{}}
        }{
    \end{array}\right)
    }
    
% Probability I - setup

% operators
\DeclareMathOperator{\Bin}{Bin}
\DeclareMathOperator{\Geo}{Geo}
\DeclareMathOperator{\Exp}{Exp}
\DeclareMathOperator{\Po}{Po}

% shortcuts
\renewcommand{\P}{\mathbb{P}}
\newcommand{\E}{\mathbb{E}}
\newcommand{\Var}{\mathbb{V}\mathrm{ar}}
\newcommand{\Cov}{\mathbb{C}\mathrm{ov}}
\newcommand{\ind}[1][A]{\mathbbm{1}_{#1}}
\newcommand{\mmid}{\,\middle|\,}
\newcommand{\cP}[2]{\P\left(#1 \mmid #2\right)}

\declaretheoremstyle[
    notefont = \normalfont\itshape,
    bodyfont = \normalfont\itshape,
    notebraces={(}{)}
]{avgstyle}
\declaretheoremstyle[
    bodyfont = \normalfont,
    notefont = \normalfont\itshape,
    spaceabove = 1em,
    spacebelow = 1em
]{defstyle}
\declaretheoremstyle[
    bodyfont = \normalfont,
    notefont = \normalfont,
    spaceabove = 1em,
    spacebelow = 1em
]{exampstyle}

\declaretheorem[style = avgstyle, numberwithin=subsection]{theorem}
\declaretheorem[numberwithin=subsection, sibling=theorem]{lemma}
\declaretheorem{proposition, corollary, remark}[style=plain, numberwithin=subsection, sibling=theorem]
\declaretheorem{definition}[style=defstyle, numbered=unless unique, numberwithin=subsection, sibling=theorem]
\declaretheorem{outline}[style=plain, numbered=no]
\declaretheorem{example}[style=exampstyle, numberwithin=subsection, sibling=theorem]

\declaretheorem[style = avgstyle, name = Problem]{problem}

\newenvironment{solution}
    {
    \begin{proof}[Solution]\renewcommand{\qedsymbol}{$\triangle$}
    }
    {
    \end{proof}
    }

% for some reason adding a new environment demands I have a new line after it otherwise it breaks.
\usepackage{expl3}

\usepackage{amssymb, amsthm, mathtools}
\usepackage[unicode,colorlinks=true,linkcolor=blue,urlcolor=magenta, citecolor=blue]{hyperref}

\usepackage[warnings-off={mathtools-colon,mathtools-overbracket}]{unicode-math}

\input{macros/common}
\input{macros/print}

\title{AnalysisTheorems}
\author{ElectrixFox}

\begin{document}
\maketitle
\input{content}
\end{document}


\title{AnalysisTheorems}
\author{ElectrixFox}

\begin{document}
\maketitle
% In this file you should put the actual content of the blueprint.
% It will be used both by the web and the print version.
% It should *not* include the \begin{document}
%
% If you want to split the blueprint content into several files then
% the current file can be a simple sequence of \input. Otherwise It
% can start with a \section or \chapter for instance.

\input{chapters/realnumbers}
\input{chapters/sequences}
\input{chapters/subsequences}
\input{chapters/series}
\input{chapters/continuity}
\input{chapters/differentiability}
\end{document}


\title{AnalysisTheorems}
\author{ElectrixFox}

\begin{document}
\maketitle
% In this file you should put the actual content of the blueprint.
% It will be used both by the web and the print version.
% It should *not* include the \begin{document}
%
% If you want to split the blueprint content into several files then
% the current file can be a simple sequence of \input. Otherwise It
% can start with a \section or \chapter for instance.

\chapter{realnumbers}\label{cha:realnums}

\begin{definition}[Bounded above]\label{def:boundabove}
  \lean{bound_above}
  \leanok
  Given a set $X$,
  $X$ is bounded above if there exists a $C$ such that for all $x \in X$ $x \leq C$.
\end{definition}

\begin{definition}[Supremum]\label{def:supremum}
  \lean{supremum}
  \leanok
  Let $X \subset \R$ be bounded above.
  A number $C \in \R$ is called a supremum of $X$ if $C$ is an upper bound of $X$ and whenever $B$ is an upper bound of $X$,
  then $C \leq B$.
\end{definition}

axiom completeness_axiom (X : Set ℝ) [Nonempty X] : bound_above X → ∃ C, supremum X C
\begin{axiom}[Completeness axiom]\label{ax:complax}
  \leanok
  \lean{completeness_axiom}
  \uses{def:supremum, def:boundabove}
  For any nonempty set $X$ bounded above there exists a supremum $C$ of $X$.
\end{axiom}

\begin{lemma}[Subset of bounded set is bounded]
  \label{lem:subsetboundbounded}
  \lean{subset_bound_bounded}
  \leanok
  \uses{def:boundabove}
  Given a set $X$ bounded above.
  For all subsets $Y \subset X$,
  $Y$ is bounded above.
\end{lemma}



\chapter{Sequences}\label{cha:sequences}

\begin{definition}[Limit of a sequence]\label{def:seqlimit}
    \lean{seq_is_limit} \leanok
    The limit of a sequence $x_n$ as $n \to \infty$ is
    \[
    \forall \varepsilon > 0, \exists N \in \N, \forall n \geq N, |x_n - l| < \varepsilon.
    \]
\end{definition}

\begin{theorem}[Limit of a sequence is unique]\label{thm:seqlimitunique}
    \lean{seq_uniquelim} \leanok
    If a sequence $x_n$ converges to $l$ and $m$,
    then $l = m$.
\end{theorem}
\begin{proof}
    \uses{def:seqlimit}
    To-Do.
\end{proof}

\begin{definition}[Sequence bounded above]\label{def:seqboundabove}
    \lean{seq_bound_above} \leanok
    A sequence $x_n$ is bounded above if ${x_n \mid n \in \N}$ is bounded above.
\end{definition}

\begin{definition}[Sequence bounded below]\label{def:seqboundbelow}
    \lean{seq_bound_below} \leanok
    A sequence $x_n$ is bounded below if ${x_n \mid n \in \N}$ is bounded below.
\end{definition}

\begin{definition}[Sequence bounded]\label{def:seqbound}
    \lean{seq_bounded} \leanok
    \uses{def:seqboundabove, def:seqboundbelow}
    A sequence $x_n$ is bounded if it is both bounded above and bounded below.
\end{definition}

\begin{lemma}[Convergent sequence bounded below has limit bounded below]\label{lem:seqboundbelow_limboundbelow}
    \lean{conv_seq_bound_below_imp_lim_bound_below} \leanok
    Let $x_n$ be a real convergent sequence,
    if $x_n$ is bounded below,
    then the limit of $x_n$ is bounded below.
\end{lemma}
\begin{proof}
    \uses{def:seqlimit, def:seqboundbelow}
    To-Do.
\end{proof}

\begin{lemma}[Convergent sequence bounded above has limit bounded above]\label{lem:seqboundabove_limboundabove}
    \lean{conv_seq_bound_above_imp_lim_bound_above} \leanok
    Let $x_n$ be a real convergent sequence,
    if $x_n$ is bounded above,
    then the limit of $x_n$ is bounded above.
\end{lemma}
\begin{proof}
    \uses{def:seqlimit, def:seqboundabove}
    To-Do.
\end{proof}

\begin{theorem}[Harmonic sequence converges]\label{thm:seq_harmonic_conv}
    \lean{harmonic_seq_conv} \leanok
    The sequence $x_n = \frac{1}{n}$ converges to zero.
\end{theorem}
\begin{proof}
    \uses{def:seqlimit, thm:archimedes}
    To-Do.
\end{proof}

\begin{theorem}[Convergent sequence is bounded]\label{thm:seq_conv_bound}
    \lean{conv_seq_is_bounded} \leanok
    If a sequence $x_n$ converges to $x$,
    then $x_n$ is bounded.
\end{theorem}
\begin{proof}
    \uses{def:seqlimit, def:seqbound}
    To-Do.
\end{proof}

\begin{corollary}[Convergent sequence is absolutely bounded]\label{cor:seq_conv_abs_bound}
    \lean{conv_seq_is_bounded_abs} \leanok
    If a sequence $x_n$ converges to $x$,
    then $x_n$ there exists a $C > 0$ such that for all $n \in \N$ $|x_n| \leq C$.
\end{corollary}
\begin{proof}
    \uses{thm:seq_conv_bound}
    To-Do.
\end{proof}

\begin{theorem}[Squeeze theorem (to zero)]\label{thm:squeeze_zero}
    \lean{seq_squeeze_zero} \leanok
    Let $x_n, y_n$ be real convergent sequences with $y_n \to 0$ as $n \to \infty$,
    and $\forall n \in \N, |x_n| \leq y_n$ then
    \[
    x_n \to 0 \text{ as } n \to \infty.
    \]
\end{theorem}
\begin{proof}
    \uses{def:seqlimit}
    To-Do.
\end{proof}

\begin{theorem}[Inverse sequence converges]\label{thm:seq_inv_conv}
    \lean{inv_seq_conv} \leanok
    For all $a \geq 0$ the sequence $x_n = \frac{1}{n ^ a}$ converges to zero.
\end{theorem}
\begin{proof}
    \uses{def:seqlimit, thm:seq_harmonic_conv, thm:squeeze_zero}
    To-Do.
\end{proof}

\begin{theorem}[Scalar multiplication (COLT)]\label{thm:seq_COLT_scalar_mult}
    \lean{seq_COLT_scalarmult} \leanok
    Let $x_n$ be a real convergent sequence with limit $l$,
    then for any $a \in \R$,
    \[
    a * x_n \to a * l \text{ as } n \to \infty.
    \]
\end{theorem}
\begin{proof}
    \uses{def:seqlimit}
    To-Do.
\end{proof}

\begin{theorem}[Addition of convergent sequences (COLT)]\label{thm:seq_COLT_add}
    \lean{seq_COLT_addition} \leanok
    Let $x_n, y_n$ be real convergent sequences with limits $l, m$ respectively,
    then
    \[
    x_n + y_n \to l + m \text{ as } n \to \infty.
    \]
\end{theorem}
\begin{proof}
    \uses{def:seqlimit}
    To-Do.
\end{proof}

\begin{lemma}[Linearity of convergent sequences (COLT)]\label{lem:seq_COLT_linear}
    \lean{seq_COLT_linearity} \leanok
    Let $x_n, y_n$ be real convergent sequences with limits $l, m$ respectively,
    then for all $a, b \in \R$,
    \[
    a * x_n + b * y_n \to a * l + b * m \text{ as } n \to \infty.
    \]
\end{lemma}
\begin{proof}
    \uses{thm:seq_COLT_add, thm:seq_COLT_scalarmult}
    This is a direct consequence of the previous two theorems.
\end{proof}

\begin{theorem}[Multiplication of convergent sequences (COLT)]\label{thm:seq_COLT_mult}
    \lean{seq_COLT_mult} \leanok
    Let $x_n, y_n$ be real convergent sequences with limits $l, m$ respectively,
    then
    \[
    x_n * y_n \to l * m \text{ as } n \to \infty.
    \]
\end{theorem}
\begin{proof}
    \uses{def:seqlimit, lem:seqlimboundabove_limboundabove, cor:conv_seq_is_bounded_abs}
    To-Do.
\end{proof}

\begin{theorem}[Division of convergent sequences (COLT)]\label{thm:seq_COLT_ratio}
    \lean{seq_COLT_ratio}
    Let $x_n, y_n$ be real convergent sequences with limits $l, m$ respectively,
    then if $m \neq 0$,
    and $\forall n \in \N, y_n \neq 0$,
    then
    \[
    \frac{x_n}{y_n} \to \frac{l}{m} \text{ as } n \to \infty.
    \]
\end{theorem}
\begin{proof}
    \uses{def:seqlimit, thm:seq_COLT_mult, thm:seq_COLT_scalarmult}
    To-Do.
\end{proof}

\begin{theorem}[Limit of bounded sequence is in the interval]\label{thm:seq_limit_in_interval}
    \lean{seq_limininterval} \leanok
    Let $x_n$ be a real convergent sequence with limit $l$,
    if $x_n \in (a, b)$ for all $n \in \N$,
    then $l \in (a, b)$.
\end{theorem}
\begin{proof}
    \uses{def:seqlimit, lem:seqboundabove_limboundabove, lem:seqboundbelow_limboundbelow}
    To-Do.
\end{proof}

\begin{lemma}[Inequality of two sequences implies limit inequality]\label{lem:seq_limineq}
    \lean{seq_limineq}
    Let $x_n, y_n$ be real convergent sequences with limits $x, y$ respectively,
    if $\forall n \in \N, x_n \leq y_n$,
    then $x \leq y$.
\end{lemma}
\begin{proof}
    \uses{def:seqlimit}
    To-Do.
\end{proof}

\begin{theorem}[Continuity of root]\label{thm:seq_cor}
    \lean{seq_sqrtcont}
    Let $x_n$ be a real convergent sequence,
    converging to $x$,
    with $x_n \geq 0$ for all $n \in \N$ then
    \[
    \lim_{n \to \infty} \sqrt{x_n} = \sqrt{x}.
    \]
\end{theorem}
\begin{proof}
    \uses{def:seqlimit, thm:seq_COLT_scalarmult, thm:seq_COLT_mult}
    To-Do.
\end{proof}

\begin{definition}[Sequence monotonic increasing]\label{def:seqmonoinc}
    \lean{seq_mono_inc} \leanok
    A sequence $x_n$ is monotonic increasing if
    \[
    \forall m \leq n, x m \leq x n.
    \]
\end{definition}

\begin{definition}[Sequence monotonic decreasing]\label{def:seqmonodec}
    \lean{seq_mono_dec} \leanok
    A sequence $x_n$ is monotonic decreasing if
    \[
    \forall m \leq n, x m \geq x n.
    \]
\end{definition}

\begin{theorem}[Bounded monotonically increasing sequence converges]\label{thm:seq_bound_monoinc_conv}
    If a sequence $x_n$ is monotonically increasing and bounded,
    then $x_n$ converges.
\end{theorem}
\begin{proof}
    \uses{def:seqlimit, def:seqmonoinc, def:seqbound}
    To-Do.
\end{proof}

\begin{theorem}[Bounded monotonically decreasing sequence converges]\label{thm:seq_bound_monodec_conv}
    If a sequence $x_n$ is monotonically decreasing and bounded,
    then $x_n$ converges.
\end{theorem}
\begin{proof}
    \uses{def:seqlimit, def:seqmonodec, def:seqbound}
    To-Do.
\end{proof}

\begin{theorem}[Bounded monotonic sequence converges]\label{thm:seq_mono_conv}
    \lean{seq_mono_bound_conv }
    If a sequence $x_n$ is monotonic and bounded,
    then $x_n$ converges.
\end{theorem}
\begin{proof}
    \uses{def:seqlimit, def:seqmonoinc, def:seqmonodec, def:seqbound}
    To-Do.
\end{proof}
\chapter{Subsequences}\label{cha:subsequences}

\begin{definition}[Extraction]\label{def:extraction}
    \lean{extraction} \leanok
    Let $\phi : \N \to \N$ be a strictly increasing sequence.
    $\phi_n$ is the extraction of $x_n$.
\end{definition}

\begin{definition}[Sequence decreasing]\label{def:seqdec}
    \lean{seq_dec} \leanok
    A sequence $x_n$ is decreasing if $\forall n \in \N$,
    \[
    x_{n + 1} \leq x_n.
    \]
\end{definition}

\begin{definition}[Sequence increasing]\label{def:seqinc}
    A sequence $x_n$ is increasing if $\forall n \in \N$,
    \[
    x_{n + 1} \geq x_n.
    \]
\end{definition}

\begin{lemma}[Subsequence of bounded sequence is bounded]\label{lem:seqboundsubseqbound}
    \lean{seq_bound_imp_subseq_bound} \leanok
    \uses{def:boundabove, def:boundbelow}
    Let $x_n$ be a bounded sequence.
    Then for any extraction $a$ of $x_n$,
    $x_{a_n}$ is also bounded.
\end{lemma}
\begin{proof}
    \uses{def:seqboundabove, def:seqboundbelow, def:seqbound}
    To-Do.
\end{proof}

\begin{lemma}[Subsequence greater than its index]\label{lem:subseq_ge_indx}
    \lean{subseq_ge_index} \leanok
    Let $x_n$ be a sequence.
    Let $a_n$ be an extraction of $x_n$.
    Then
    \[
    \forall j \in \N, a_j \geq j.
    \]
\end{lemma}
\begin{proof}
    \uses{def:extraction}
    To-Do.
\end{proof}

\begin{lemma}[All subsequences of a convergent sequence converge to its limit]\label{lem:subseqconvtoseqlim}
    \lean{subseq_conv_to_seq_limit} \leanok
    Let $x_n$ be a convergent sequence with limit $l$.
    Then for any extraction $a_n$ of $x_n$,
    the subsequence $x_{a_n}$ converges to $l$. 
\end{lemma}
\begin{proof}
    \uses{def:seqlimit, lem:subseq_ge_indx}
    To-Do.
\end{proof}

\begin{lemma}[Every sequence has a monotonic subsequence]\label{lem:subseqmonoincordec}
    \lean{seq_contsub_inc_or_dec}
    Let $x_n$ be a sequence.
    For any subsequence $a_n$ of $x_n$,
    $x_{a_n}$ is either increasing or decreasing.
\end{lemma}
\begin{proof}
    To-Do.
\end{proof}

\begin{theorem}[Bolzano-Weierstrass]\label{thm:bolzano-weierstrass}
    \lean{subseq_BolzanoWeierstrass} \leanok
    Every bounded sequence has a convergent subsequence.
\end{theorem}
\begin{proof}
    \uses{lem:subseqmonoincordec, def:seqlimit, thm:seq_mono_conv, lem:seqboundsubseqbound}
    To-Do.
\end{proof}
\chapter{Series}\label{cha:series}

\section{Definitions and Basic Properties}

\begin{definition}[Partial sums]\label{def:partial_sums}
  \lean{seq_partial_sums} \leanok
  The partial sums of a sequence $x_n$ is defined as
  \[
  s_n = \sum_{i = 0}^{n} x_i.
  \]
\end{definition}

\begin{definition}[Convergent series]\label{def:convergent_series}
  \lean{sum_is_limit} \leanok
  A series $\sum_{i = 0}^{\infty} x_i$ converges to $l$ if the sequence of partial sums $s_n$ converges to $l$.
\end{definition}

\begin{lemma}\label{lem:sum_tail_conv}
    \lean{sum_tail_conv} \leanok
    \uses{def:partial_sums, def:convergent_series}
    A series $\sum_{i = 0}^{\infty} x_i$ converges to $l$ if and only if the tail series $\sum_{i = N}^{\infty} x_i$ converges.
\end{lemma}
\begin{proof}
    \uses{def:convergent_series}
    To-Do.
\end{proof}

\begin{lemma}\label{lem:sumifconv_seqconvtozero}
    If $\infsumo a_k$ is convergent,
    then
    \[
    \liminfty[k] a_k = 0.
    \]
\end{lemma}
\begin{proof}
    \uses{def:convergent_series, def:seqlimit, lem:seq_COLT_linear}
    Let $S_n = \sum_{k = 0}^{n}a_k$,
    then $\liminfty S_n = s$ or some $s \in \R$.
    Write $a_n = S_n - S_{n - 1}$ for $n \geq 1$.
    By COLT
    $\lim_{n \to \infty}a_n = s - s = 0$.
\end{proof}

\begin{lemma}[Harmonic Series]\label{lem:harmsumdiv}
    The series
    \[
    \infsum\frac{1}{n}
    \]
    diverges.
\end{lemma}
\begin{proof}
    Let $S_n = \sum_{k = 1}^{n}\frac{1}{k}$ then
    \[
    S_{2 ^ n} \geq 1 + \frac{n}{2}
    \]
    so the sequence of partial sums diverges.
    Thus the series diverges.
\end{proof}

\begin{theorem}[Addition of series (COLT)]\label{thm:sum_COLT_add}
    Let $\infsumo a_k$ and $\infsumo b_k$ be convergent series with limits $a, b$ respectively.
    Then
    \[
    \infsumo (a_k + b_k) = a + b.
    \]
\end{theorem}
\begin{proof}
    \uses{def:convergent_series, def:partial_sums, thm:seq_COLT_add}
    Use COLT for sequences on the partial sums.
\end{proof}

\begin{theorem}[Scalar multiplication of series (COLT)]\label{thm:sum_COLT_scalarmult}
    Let $\infsumo a_k$ be a convergent series with limit $a$.
    Then for any $c \in \R$,
    \[
    \infsumo (c * a_k) = c * a.
    \]
\end{theorem}
\begin{proof}
    \uses{def:convergent_series, def:partial_sums, thm:seq_COLT_scalar_mult}
    Use COLT for sequences on the partial sums.
\end{proof}

\section{Convergence Criteria}

\begin{theorem}[Comparison Test]\label{thm:sum_comptest}
    Let $N \in \N$ let $(a_k)_{k \in \N_0}$, $(b_k)_{k \in \N_0}$ sequences with
    \[
    0 \leq a_k \leq b_k \text{ for } k \geq N.
    \]
    \begin{enumerate}[label = (\alph*)]
        \item If $\infsum[k = 0]b_k$ is convergent,
        then $\infsum[k = 0]a_k$ is convergent.
        \item If $\infsum[k = 0]a_k$ is divergent,
        then $\infsum[k = 0]b_k$ is divergent.
    \end{enumerate}
\end{theorem}
\begin{proof}
    \uses{def:convergent_series, def:partial_sums, thm:seq_COLT_add, lem:sum_tail_conv}
    Use $\infsum[k = 0]a_k$ is convergent if and only if $\infsum[k = N]a_k$ is convergent.

    For $n \geq N$ let $s_n = \sum_{k = N}^{n}a_k$ and $t_n = \sum_{k = N}^{n}b_k$.
    Theses are monotonically increasing.

    If $\infsum[k = 0]b_k$ is convergent so is $(t_n)_{n \geq N}$ to $t$.
    \[
    s_n \leq t_n \leq t.
    \]
    The sequence $s_n$ is monotonically increasing by $t$,
    so is convergent by Thm 2.16.

    (b)
    Since $s_n$ is monotonically increasing,
    divergence means $(s_n)$ is unbounded,
    but then $(t_n)$ is also unbounded.
\end{proof}

\begin{theorem}\label{thm:suminvseqconv}
    Let $\alpha \in \R$.
    Then
    \[
    \infsum\frac{1}{k ^ \alpha}\text{ is convergent, if and only if $\alpha > 1$}.
    \]
\end{theorem}
\begin{proof}
    \uses{thm:sum_comptest, thm:seq_inv_conv, lem:harmsumdiv}
    To-Do.
\end{proof}

\begin{definition}\label{def:alternating_series}
    A series $\infsum a_k$ is called alternating,
    if
    \[
    a_{2k} \geq 0\text{ and } a_{2k - 1} \leq 0\text{ for all $k \in \N$}
    \]
    or if 
    \[
    a_{2k} \leq 0\text{ and } a_{2k - 1} \geq 0\text{ for all $k \in \N$}.
    \]
\end{definition}

\begin{theorem}[Alternating Sign Test]\label{thm:sum_alternating_sign_test}
    Let $\dseq[k]{a}$ be a monotonically decreasing sequence of positive numbers with $\lim_{k \to \infty}a_k = 0$.
    Then the alternating series $\infsum(-1) ^ {k + 1}a_k$ is convergent.

    For the sequence of partial sums $\dseq{s}$ we have
    \[
    S_2 \leq S_4 \leq \dotsi \leq S_{2n} \leq \dotsi \leq \infsum(-1) ^ {k + 1}a_k \leq \dotsi \leq S_{2n - 1} \leq \dotsi \leq S_3 \leq S_1
    \]
    and
    \[
    \left|S_n - \infsum(-1) ^ {k + 1}a_k\right| \leq a_{n + 1}.
    \]
\end{theorem}
\begin{proof}
    \uses{def:convergent_series, def:partial_sums, thm:sum_COLT_add}
    To-Do.
\end{proof}

\section{Absolute Convergence}

\begin{definition}[Absolute convergence]\label{def:sum_abs_convergence}
    Let $\displaystyle\infsum[k = 1]a_k$ be a series,
    we call it absolutely convergent,
    if $\displaystyle\infsum[k = 1]|a_k|$ is convergent.
\end{definition}

\begin{theorem}[Absolute convergence implies convergence]\label{thm:sum_abs_convergence_implies_convergence}
    Let $\infsum[k = 1]a_k$ be absolutely convergent.
    Then the series is convergent.
\end{theorem}
\begin{proof}
    \uses{def:convergent_series, def:partial_sums, thm:sum_COLT_add}
    We have $\infsum[k = 1]|a_k|$ is convergent.
    By COLT
    \[
    \infsum[k = 1]2|a_k|\text{ is convergent}.
    \]
    We have $-|a_k| \leq a_k \leq |a_k|$.
    So $0 \leq |a_k| + a_k \leq 2|a_k|$.
    By the comparison test,
    \[
    \infsum[k = 1](a_k + |a_k|)
    \]
    is convergent.
    By COLT
    \[
    \infsum[k = 1]a_k = \infsum[k = 1]((a_k + |a_k|) - |a_k|)
    \]
    is convergent.
    Additionally,
    \[
    \infsum[k = 1]a_k \leq \infsum[k = 1]|a_k|
    \]
    we have
    \[
    \sum_{k = 1}^{n}a_k \leq \sum_{k = 1}^{n}|a_k|.
    \]
\end{proof}

\begin{theorem}[Ratio Test]\label{thm:sum_ratio_test}
    Let $\dseq[k]{a}$ be a sequence with $a_k \neq 0$ for all $k \in \N$ except finitely many.
    \begin{enumerate}[label = (\alph*)]
        \item If $\displaystyle\lim_{k \to \infty}\left|\frac{a_{k + 1}}{a_k}\right| < 1$,
        then the series $\displaystyle\infsum[k = 1]$ is absolutely convergent.
        \item If $\displaystyle\lim_{k \to \infty}\left|\frac{a_{k + 1}}{a_k}\right| > 1$,
        then the series $\displaystyle\infsum[k = 1]a_k$ is divergent.
    \end{enumerate}
\end{theorem}
\begin{proof}
    \uses{def:convergent_series, def:partial_sums, thm:seq_COLT_add}
    To-Do.
\end{proof}

\begin{theorem}[Root test]\label{thm:sum_root_test}
    For a sequence $\dseq[k]{a}$ set
    \[
    a = \limsup_{k \to \infty}\sqrt[k]{|a_k|}
    \]
    \begin{enumerate}[label = (\alph*)]
        \item If $a < 1$,
        then $\displaystyle\infsum[k = 1]a_k$ is absolutely convergent.
        \item If $a > 1$,
        then $\displaystyle\infsum[k = 1]a_k$ is divergent.
    \end{enumerate}
\end{theorem}
\begin{proof}
    \uses{def:convergent_series, def:partial_sums, thm:seq_COLT_add}
    Assume $a < 1$.
    Then for all but finitely many $k$ we have
    \[
    \sqrt[k]{|a_k|} \leq q < 1
    \]
    use $q = \frac{a + 1}{2}$.
    So for all $k \geq n_0$ we have $|a_k| \leq q ^ k$.
    By comparison test with the convergent geometric series
    \[
    \infsum[k = 1]q ^ k
    \]
    we get absolute convergence.

    Assume $a > 1$,
    then for all but finitely many $k$,
    $\sqrt[k]{|a_k|} \geq q > 1$.
    hence for all $k \geq n_1$ $|a_k| \geq q ^ k$.
    By the comparison test with the diverging geometric series
    \[
    \infsum[k = 1]q ^ k
    \]
    we get divergence.
\end{proof}

\begin{definition}[Conditionally convergent series]\label{def:cond_sum}
    Let $\infsum a_k$ be a series.
    We say this series is conditionally convergent,
    if it is convergent,
    but not absolutely convergent.
\end{definition}
\chapter{Continuity}\label{cha:continuity}

\begin{definition}[Open interval]\label{def:openinterval}
    \[
    x \in (a, b) \iff a < x < b
    \]
    \textit{$a = -\infty, b = \infty$ is fine}.
\end{definition}

\begin{definition}[Open set]\label{def:openset}
    Given $X \subseteq \R$. For all $c \in X$ there exists an open interval in $X$ containing $c$.
    
    That is,
    there exists $\delta > 0$,
    such that
    \[
    (c - \delta, c + \delta) \subseteq X.
    \]
\end{definition}

\begin{definition}[Interior point]\label{def:interiorpoint}
    An interior point $c \in X$ if there exists $(c - \delta, c + \delta) \subset X$.
\end{definition}

\begin{definition}[Closed interval]\label{def:closedinterval}
    \[
    x \in [a, b] \iff a \leq x \leq b
    \]
    ($a, b \in \R$).
\end{definition}

\begin{lemma}\label{lem:limitliesinab}
    $(x_n) \in [a, b]$ converging with $\liminfty x_n = L$.
    Then $L \in [a, b]$.
\end{lemma}
\begin{proof}
    Assume $L \notin [a, b]$;
    take $\varepsilon = \min\{|L - b|, |L - a|\}$.
    Say $L > b$.
    For all $x_n$ we have
    \begin{align*}
        |x_n - L| &= |x_n - b + b - L| \\
        &= (b - x_n) + (L - b) \\
        &> \varepsilon.
    \end{align*}
    Contradiction to $\liminfty x_n = L$.
\end{proof}

\begin{theorem}[Bolzano-Weierstrass]\label{thm:cont_bolzanoweierstrass}
    $(x_n) \in [a, b]$,
    with $a, b \in \R$,
    has a converging subsequence converging in $[a, b]$.
\end{theorem}
\begin{proof}
    $(x_n)$ are bounded which by \autoref{thm:seq_bolzanoweierstrass} has a convergent subsequence and by \autoref{lem:limitliesinab}.
\end{proof}

\begin{definition}[Compact interval]\label{def:compactinterval}
    Call $[a, b]$
    ($a, b \in \R$)
    a compact interval if the interval is bounded and closed.
\end{definition}

\section{Limits of functions}

\begin{definition}[Limit of a function]\label{def:limitfunction}
    Let $f : (a, b) \to \R$ be a function.
    Let $c \in (a, b)$ and $f$ is possibly not defined at $c$.
    We say
    \[
    \lim_{x \to c}f(x) = L
    \]
    if for all $\varepsilon > 0$ there exists a $\delta > 0$ such that
    \[
    |f(x) - L| < \varepsilon
    \]
    for all $x \neq c$ with
    \[
    |x - c| < \delta.
    \]
    We also write $f(x) \to L$ as $x \to c$.
\end{definition}

\begin{definition}[Limit from the right]\label{def:limitfromright}
    \[
    \lim_{x \to c ^ {+}}f(x) \text{ same but all } x > c.
    \]
\end{definition}


\begin{definition}[Limit from the left]\label{def:limitfromleft}
    \[
    \lim_{x \to c ^ {-}}f(x) \text{ same but all } x < c.
    \]
\end{definition}

\begin{definition}[Infinite limit]\label{def:infinlimit}
    $\liminfty[x] f(x) = L$:
    for all $\varepsilon > 0$ there exists a $k \in \R$ such that
    \[
    |f(x) - L| < \varepsilon\qquad\text{for all } x > k.
    \]
\end{definition}

\begin{proposition}[Limit of a function and sequences]\label{prop:limitfunctionandseq}
    \[
    \lim_{x \to c}f(x) = L
    \]
    \[
    \iff
    \]
    for all sequences $(x_n)$ with $\liminfty x_n = c$ have $\liminfty f(x_n) = L$.
\end{proposition}
\begin{proof}
    \uses{def:limitfunction, def:seqlimit}
    "$\implies$".
    Assume $\lim_{x \to c}f(x) = L$.
    Take $(x_n) \in (a, b)$ $(x_n \neq c)$ with $\liminfty x_n = c$.
    Take $\varepsilon > 0$.
    Need an $N$ such that
    \[
    |f(x_n) - L| < \varepsilon
    \]
    for all $n \geq N$.
    We know there exists a $\delta > 0$ such that
    \begin{equation}\label{eq:1}
        |f(x) - L| < \varepsilon
    \end{equation}
    for all $|x - c| < \delta$ $(x \neq c)$.
    Since $\liminfty x_n = c$ there exists an $N$ such that $|x_n - c| < \delta$ for all $n \geq N$.
    By \eqref{eq:1} $|f(x_n) - L| < \varepsilon$ for all $n \geq N$.

    "$\impliedby$".
    By contrapositive.
    Assume $\lim_{x \to c}f(x) \neq L$
    (or does not exist).
    Need to find a sequence $x_n$ where $\liminfty x_n = c$ but $\liminfty f(x_n) \neq L$
    (or does not exist).
    Hence there exists a $\varepsilon > 0$ such that for all $\delta > 0$ such that there exists an $x$ with $|x - c| < \delta$ but $|f(x) - L| \geq \varepsilon$.
    Take the "bad" $\varepsilon > 0$.
    Take $\delta = 1 / n$,
    get an $x = x_n$ with $|x_n - c| < \delta = \frac{1}{n}$ but $|f(x_n) - L| \geq \varepsilon$.
    \[
    \liminfty x_n = c
    \]
    but
    \[
    \liminfty f(x_n) \neq L.
    \]
    This completes the proof by the contrapositive.
\end{proof}

\begin{lemma}[Linear combination of limits]\label{lem:linearcombinationlimits}
    We have $\lim_{x \to c}f(x) = L_1$ and $\lim_{x \to c}g(x) = L_2$. Then
    \[
    \lim_{x \to c}(af(x) + bg(x)) = aL_1 + bL_2.
    \]
\end{lemma}
\begin{proof}
    Using the previous proposition and applying COLT for sequences.
\end{proof}

\begin{lemma}[Product of limits]\label{lem:productlimits}
    We have $\lim_{x \to c}f(x) = L_1$ and $\lim_{x \to c}g(x) = L_2$. Then
    \[
    \lim_{x \to c}(f(x)g(x)) = L_1L_2.
    \]
\end{lemma}
\begin{proof}
    Take $x_n \to c$. By COLT for sequences,
    \[
    \liminfty\left[f(x_n)g(x_n)\right] = \lim_{x \to c}f(x_n) \cdot \liminfty g(x_n) = L_1L_2.
    \]
\end{proof}

\begin{lemma}[Quotient of limits]\label{lem:quotientlimits}
    We have $\lim_{x \to c}f(x) = L_1$ and $\lim_{x \to c}g(x) = L_2$ with $L_2 \neq 0$. Then
    \[
    \lim_{x \to c}\left(\frac{f(x)}{g(x)}\right) = \frac{L_1}{L_2}.
    \]
\end{lemma}
\begin{proof}
    Using the previous proposition and applying COLT for sequences.
\end{proof}

\begin{proposition}[Squeezing]\label{prop:fun_squeeze}
    Assume $f(x) \leq g(x) \leq h(x)$.
    For all $x$ in a neighbourhood\footnote{Close to $c$.} of $c$ with
    \[
    \lim_{x \to c}f(x) = \lim_{x \to c}h(x) = L.
    \]
    Then
    \[
    \lim_{x \to c}g(x) = L.
    \]
\end{proposition}

\section{Continuous functions}

\begin{definition}[Continuity at a point]\label{def:continuityatpoint}
    $f : X \to \R$,
    $X = (a, b)$
    $c \in (a, b)$.
    Call $f(x)$ continuous at $x = c$ if
    \[
    \lim_{x \to c}f(x) = f(c).
    \]
    For all $\varepsilon > 0$,
    there exists $\delta > 0$ such that
    \[
    |f(x) - f(c)| < \varepsilon
    \]
    for all $x$ with
    \[
    |x - c| < \delta.
    \]
\end{definition}

\begin{proposition}[Continuity and limits]\label{prop:continuityandlimits}
    $f(x)$ is continuous at $x = c$ if and only if
    \[
    \liminfty f(x_n) = f\left(\liminfty x_n\right).
    \]
\end{proposition}
\begin{proof}
    \uses{def:continuityatpoint, def:seqlimit}
    To-do.
\end{proof}

\begin{proposition}[Continuity of composition]\label{prop:continuityofcomposition}
    Assume $f$ is continuous at $c \in X$ and $g$ is continuous at $f(c) \in Y$.
    Then $g \circ f(x)$ is continuous at $x = c$.
\end{proposition}
\begin{proof}
    Use the sequence criterion:
    take $x_n \in X$ with $\liminfty x_n = c$.
    Need $\liminfty g \circ f(x_n) = g(f(c))$.
    
    Set $y_n = f(x_n)$ since $f$ is continuous at $c$ we have that $\liminfty f(x_n) = \liminfty y_n = f(c)$,
    this sequence,
    $f(x_n)$,
    is in $Y$.
    Since $g$ is continuous at $f(c)$ we have $\liminfty g(y_n) = \liminfty g(f(c)) = \liminfty g(f(x_n))$.
\end{proof}

\section{Great Theorems}
\begin{theorem}[Intermediate Value Theorem]\label{thm:IVT}
    $f : [a, b] \to \R$ continuous.
    with $f(a) < f(b)$
    (say).
    Pick $d \in [f(a), f(b)]$;
    $f(a) \leq d \leq f(b)$.
    Then there exists a $c \in [a, b]$
    (not necessarily unique)
    such that $f(c) = d$.
\end{theorem}
\begin{proof}
    Pick $d$,
    assume $d < f(b)$
    (otherwise can pick $c = b$).
    
    Define the set
    \[
    X \coloneqq \{x \in [a, b]; f(x) \leq d\}.
    \]
    $X \neq \emptyset$,
    since $a \in X$ and bounded as a subset of $[a, b]$.
    Hence has a supremum,
    $c$.
    (By term $1$)
    exists a sequence $x_n \in X$ such that $\liminfty x_n = c$.
    $x_n \in X \subseteq [a, b]$ hence $c = \liminfty x_n \in [a, b]$.
    By continuity $\liminfty f(x_n) = f\left(\liminfty x_n\right) = f(c)$.

    Claim:
    $f(c) = d$.

    Assume not,
    i.e. $f(c) < d$\footnote{Since $f(c) = \liminfty f(x_n) \in X$ hence $\liminfty f(x_n) \leq d$}.
    Then by problem sheet $1$
    (this term)
    Q7,
    there exists a
    (small)
    neighbourhood $(c - \delta, c + \delta) \in (a, b)$ such that $f(x) < d$ for all $x \in (c - \delta, c + \delta)$.

    In particular, $f(c + \delta / 2) < d$ so $c + \delta / 2 \in X$
    but $c < c + \delta / 2$ but $c = \sup{X}$ contradiction!

    So $f(c) =  d$.
\end{proof}

\begin{corollary}\label{cor:IVT_imagecont}
    $f : I \to \R$ continuous on an interval $I$.
    Then the image $f(I)$ is also an interval.
\end{corollary}
\begin{proof}
    An interval $J$ is a set such that whenever $x < y \in J$,
    then all numbers in between are also in $J$.
    Now apply Intermediate Value Theorem.

    \textit{Use $x = f(a), y = f(b)$ and apply IVT.}
\end{proof}

\begin{theorem}\label{thm:IVT_maxmin}
    $f : [a, b] \to \R$ is continuous.
    Then $f$ takes minimum and maximum on $[a, b]$.
\end{theorem}
\begin{proof}
    Only do maximum.

    Step $1$.
    
    $f$ is bounded above,
    on $[a, b]$.

    Say it did,
    then given $n \in \N$,
    exists $x_n \in [a, b]$ such that $f(x_n) > n$ by Bolzano-Weierstrass there exists a convergent subsequence $x_{n_i}$ with limit $c \in [a, b]$,
    here we use closed interval.
    So $f(n_i) \to f(c) \in \R$ with $f(n_i) > n_i$,
    at some point $n_i > c$.

    Hence $\sup\{f(a, b)\}$ exists in $\R$,
    $M = \sup\{f(a, b)\}$.
    (By term $1$)
    there exists a sequence $y_n \in f([a, b])$ such that $\liminfty y_n = M$,
    but $y_n = f(x_n)$ by continuity $\liminfty f(x_n) = M$.

    $x_n$ might not converge but by Bolzano-Weierstrass will have a converging subsequence in $[a, b]$ so $\lim x_{n_i} = c \in [a, b]$.

    Then $f(c) = f(\lim x_{n_i}) = \lim f(x_{n_i}) = \lim y_{n_i} = M$.

    Together with the Intermediate Value theorem we get the image of a continuous function on a compact interval is again a compact interval.
\end{proof}

\begin{definition}[Continuity on a set]\label{def:continuityonset}
    Continuity on a set $X$,
    for all $c \in X$ and for all $\varepsilon > 0$,
    there exists $\delta > 0$ for all $x \in X$ with
    \[
    |x - c| < \delta \implies |f(x) - f(c)| < \varepsilon.
    \]
\end{definition}

\begin{definition}[Uniform continuity]\label{def:uniformcontinuity}
    $f : X \to \R$ is uniform continuous if for all $\varepsilon > 0$ there exists $\delta > 0$ such that for all $x, y \in X$ with
    \[
    |x - y| < \delta \implies |f(x) - f(y)| < \varepsilon.
    \]
    \textit{In other words}
    \[
    \forall \varepsilon > 0, \exists \delta > 0 \text{ s.t. } \forall x, y \in X, |x - y| < \delta \implies |f(x) - f(y)| < \varepsilon.
    \]
\end{definition}

\begin{theorem}\label{thm:contoncompactisuniform}
    $f : [a, b] \to \R$ continuous on a compact interval.
    Then $f$ is uniformly continuous.
\end{theorem}
\begin{proof}
    Assume not.
    There exists $\varepsilon > 0$ such that for all $\delta > 0$ there exists $x, y \in X$ with
    \[
    |x - y| < \delta
    \]
    but
    \[
    |f(x) - f(y)| \geq \varepsilon.
    \]
    Take such a "bad" $\varepsilon > 0$.
    So for $\delta = \delta_n = \frac{1}{n}$,
    has $x_n, y_n \in [a, b]$ with $|x_n - y_n| < \delta$ but $|f(x_n) - f(y_n)| \geq \delta$.
    By Bolzano-Weierstrass for a converging subsequence $(x_{n_i})$ of the $(x_n)$
    (since $x_n \in [a, b]$)
    say $\lim x_{n_i} = x ^ {*} \in [a, b]$.

    Claim:
    also $\lim y_{n_i} = x ^ {*}$.
    Indeed,
    \begin{align*}
        |x ^ {*} - y_{n_i}| &= |x ^ {*} - x_{n_i} + x_{n_i} - y_{n_i}| \\
        &\leq |x ^ {*} - x_{n_i}| + |x ^ {*} - y_{n_i}| \\
        &\to 0 + 0 = 0
    \end{align*}
    Squeezing gives the claim.

    Claim:
    \[
    \lim f(x_{n_i}) - f(y_{n_i}) = 0.
    \]
    Indeed
    \begin{align*}
        |f(x_{n_i}) - f(y_{n_i})| &= |f(x_{n_i}) - f(x ^ {*}) + f(x ^ {*}) - f(y_{n_i})| \\
        &\leq |f(x_{n_i}) - f(x ^ {*})| + |f(x ^ {*}) - f(y_{n_i})| \\
        &\to 0 + 0 = 0
    \end{align*}
    by continuity of $f$ and $x_{n_i}, y_{n_i} \to x ^ {*}$.

    So for $n_i$ sufficiently large
    \[
    |f(x_{n_i}) - f(y_{n_i})| < \varepsilon
    \]
    contradiction!
\end{proof}

\section{Inverse functions}
Assume $f : X \to \R$ is injective so the inverse function $f ^ {-1} : f(x) \to \R$ exists,
$f(x) = Y$.

Principle question:

if $f$ is "nice"
(e.g. continuous)
is the inverse also nice?

\begin{theorem}\label{thm:inversefunction}
    Let $f : I \to \R$ be a continuous function on an interval $I$,
    and injective
    (1-1)
    so $f(I) = J$ is also an interval and the inverse function $f ^ {-1} : J \to I$ is also continuous.
\end{theorem}
\begin{proof}
    One of the key steps:
    if $f$ is continuous and 1-1.
    Then $f$ is either strictly monotonically increasing or decreasing.
\end{proof}
\chapter{Differentiability}\label{cha:differentiability}

\begin{definition}[Derivative of a function]\label{def:differentiable}
    $f : X \to \R$
    ($X$ open).
    We say that $f$ is differentiable at a point $c \in X$ if
    \[
    \lim_{x \to c}\frac{f(x) - f(c)}{x - c}
    \]
    exists.
    If so,
    we write $f'(c)$ for the limit.
\end{definition}

\begin{lemma}[First order Taylor]\label{lem:firstordertaylor}
    $f : X \to \R$,
    $f$ is differentiable at $c$ if and only if there exists a constant $n \in \R$ and a function $r(x)$ on $X$ such that
    \begin{equation}\label{eq:2}
        f(x) = f(c) + m(x - c) + r(x)(x - c)
    \end{equation}
    with $r(x)$ is continuous at $c$ and $\lim_{x \to c} r(x) = r(c) = 0$.
    In that case $m = f'(c)$.
\end{lemma}
\begin{proof}
    \uses{def:differentiable}
    "$\implies$":
    Set $m = f'(c)$ and
    \[
    r(x) \coloneqq \begin{cases}
        \frac{f(x) - f(c) - m(x - c)}{x - c} & x \neq c, \\
        0 & x = c.
    \end{cases}
    \]
    \eqref{eq:2} holds by construction.
    Need to show
    \[
    \lim_{x \to c}r(x) = 0 = r(c),
    \]
    \begin{align*}
        \lim_{x \to c}\left(\frac{f(x) - f(c)}{x - c} - m\right) &= \lim_{x \to c}\left(\frac{f(x) - f(c)}{x - c} - f'(c)\right) = 0.
    \end{align*}

    "$\impliedby$":
    \begin{align*}
        0 &= r(c) \\
        &= \lim_{x \to c}r(x) \\
        &= \lim_{x \to c}\frac{f(x) - f(c) - m(x - c)}{x - c} \\
        &= \lim_{x \to c}\left(\frac{f(x) - f(c)}{x - c} - m\right).
    \end{align*}
    Only way this is possible if $\lim_{x \to c}\frac{f(x) - f(c)}{x - c}$ exists and is equal to $m$.
\end{proof}

\begin{proposition}[Continuity of differentiable functions]\label{prop:diffcont}
    $f : X \to \R$ as before.
    Then if $f$ is differentiable at $x = c$,
    then $f(x)$ is also continuous at $x = c$.
\end{proposition}
\begin{proof}
    Assume $f$ is differentiable at $x = c$.
    Then $f(x) - f(c) = (x - c) \frac{f(x) - f(c)}{x - c} \xrightarrow[x \to c]{} 0 \cdot f'(c) = 0$.
    So $\lim_{x \to c}f(x) = f(c)$,
    that is exactly continuity.
\end{proof}

\begin{theorem}\label{thm:sum_diff_atc}
    $f, g$ are differentiable at $x = c$.
    Then $f(x) + g(x)$ and $\alpha f(x)$ are differentiable at $x = c$.
    With
    \[
    (f + g)'(c) = f'(c) + g'(c)
    \]
    and
    \[
    (\alpha f)'(c) = \alpha f'(c).
    \]
    Also the product $f(x)g(x)$ with
    \[
    (f(x)g(x))'(c) = f'(c)g(c) + f(c)g'(c).
    \]
    Assume $f(c) \neq 0$.
    Then $\frac{1}{f(x)}$ is defined in a open neighbourhood around $x = c$ and is differentiable with
    \[
    \left(\frac{1}{f(c)}\right)' = \frac{-f'(c)}{f ^ 2(c)}
    \]
\end{theorem}
\begin{proof}
    For the product.
    Write
    \[
    f(x) = f(c) + (x - c)f_1(x)
    \]
    \[
    g(x) = g(c) + (x - c)g_1(x)
    \]
    with $f_1, g_1$ continuous at $x = c$ and $\lim_{x \to c}f_1(x) = f'(c)$ and $\lim_{x \to c}g_1(x) = g'(c)$.
    Then
    \begin{align*}
        f(x)g(x) &= (f(c) + (x - c)f_1(x))(g(c) + (x - c)g_1(x)) \\
        &= f(c)g(c) + (x - c)(f_1(x)g(c) + f(c)g_1(x) + (x - c)f_1(x)g_1(x)) \\
        \intertext{with $x = c$}
        &= f'(c)g'(c) + f(c)g'(c) + 0.
    \end{align*}
\end{proof}

\begin{theorem}[Chain rule]\label{thm:chainrule}
    $g : X \to Y \subseteq \R$,
    $f : Y \to \R$,
    $X, Y$ open,
    $g$ is differentiable at $x = c$,
    $f$ is differentiable at $y = d = g(c)$.

    Then the composition
    \[
    f \circ g(x) : X \to \R
    \]
    is differentiable at $x = c$ and
    \[
    (f \circ g)'(c) = g'(c)f'(g(c)).
    \]
\end{theorem}
\begin{proof}
    $g$ differentiable at $c$:
    \[
    g(x) = g(c) + g_1(x)(x - c)
    \]
    continuous at $c$ and $g_1(c) = \lim_{x \to c}g_1(x) = g'(c)$.

    $f$ differentiable at $g(c) = d$:
    \[
    f(y) = f(g(c)) + f_1(y)(y - g(c)).
    \]
    So
    \begin{align*}
        f \circ g(x) &= f(g(x)) \\
        &= f(g(c)) + f_1(g(x))(g(x) - g(c)) \\
        &= f(g(c)) + f_1(g(x))[g(c) + g_1(x)(x - c) - g(c) ^ 2] \\
        &= f(g(c)) + f_1(g(x))g_1(x)(x - c)
        \intertext{have $h(x) = f \circ g(x)$,
        $h_1(x) = f_1(g(x))g_1(x)$}
        &= h(c) + h_1(x)(x - c)
    \end{align*}
    since $f_1$ is continuous at $g(c)$ and $g_1$ is continuous at $c$,
    \begin{align*}
        \lim_{x \to c}h_1(x) &= f_1(g(c))g_1(c)
        \intertext{by continuity}
        &= f'(g(c))g'(c).
    \end{align*}
\end{proof}

\begin{lemma}\label{lem:e_ineq}
    \[
    x \leq e ^ x - 1 \leq \frac{x}{1 - x}\qquad(x < 1).
    \]
\end{lemma}
\begin{proof}
    To-Do.
\end{proof}

\begin{theorem}\label{thm:diff_inv_func_rule}
    $f : I \to \R$
    ($I$ an interval)
    continuous and differentiable at $x = c$,
    and $f'(c) \neq 0$.
    Assume $f$ is $1$-$1$
    (invertible).
    So
    \[
    f ^ {-1} = g : \underset{= f(I)}{Y} \to \R
    \]
    exists and is differentiable at $y = d = f(c)$ and
    \[
    (f ^ {-1})'(d) = \frac{1}{f'(f ^ {-1}(d))} = \frac{1}{f'(c)}.
    \]
\end{theorem}
\begin{proof}
    Simple case.
    Assume you knew that the inverse function $f ^ {-1}$ is differentiable.
    Then $f ^ {-1} \circ f(x) = x$ by the chain rule
    \[
    (f ^ {-1})'(f(x))f'(x) = 1
    \]
    \[
    (f ^ {-1}(f(x)))' = \frac{1}{f'(x)}
    \]
    write $x = f ^ {-1}(y)$,
    \[
    (f ^ {-1})'(y) = \frac{1}{f'(f ^ {-1}(y))}.
    \]
\end{proof}

\begin{proposition}\label{prop:diff_max_or_min}
    If $f$ is differentiable at $c$ and it has a local maximum or a local minimum at $C$,
    then $f'(c) = 0$.
\end{proposition}
\begin{proof}
    $f'(c) = \lim_{x \to c}\frac{f(x) - f(c)}{x - c}$.
    If $x > c$,
    but $x$ is near $c$,
    then $f(x) \leq f(c)$ since $c$ is a local maximum.
    In particular $\frac{f(x) - f(c)}{x - c} \leq 0$.
    Similarly,
    for $x < c$,
    $\frac{f(x) - f(c)}{x - c} \geq 0$ so $\lim_{x \to c ^ {+}}\frac{f(x) - f(c)}{x - c} \leq 0, \lim_{x \to c ^ {-}}\frac{f(x) - f(c)}{x - c} \geq 0 \implies f'(c) = 0$.
\end{proof}

\begin{theorem}[Rolle's Theorem]\label{thm:rolles}
    Let $f : [a, b] \to \R$ be continuous and differentiable on $(a, b)$,
    and suppose $f(a) = f(b)$.
    Then there exists $c \in (a, b)$ with $f'(c) = 0$.
\end{theorem}
\begin{proof}
    A continuous function on a closed interval attains a maximum and a minimum.
    So there is a $c \in [a, b]$ with $f(c) \geq f(x)$ for all $x \in [a, b]$,
    $d \in [a, b]$ with $f(d) \leq f(x)$ for all $x \in [a, b]$.
    If $c \in (a, b)$,
    we get $f'(c) = 0$ by last result.
    If $c = a$ or $c = b$.
    Then look at minimum $d$ if $f \in (a, b)$ we can use the last result again $f'(d) = 0$.
    If $d = a$ or $d = b$,
    then $f(d) = f(c)$ and the whole function is constant.
    Then $f'(x) = 0$ for all $x \in (a, b)$.
\end{proof}

\begin{theorem}[Mean Value Theorem]\label{thm:mean_value_thm}
    Let $f : [a, b] \to \R$ be continuous and differentiable on $(a, b)$.
    Then there exists a $c \in (a, b)$ such that $f'(c) = \frac{f(b) - f(a)}{b - a}$.
\end{theorem}
\begin{proof}
    $g(x) = f(x) - \frac{f(b) - f(a)}{b - a}(x - a)$.
    Then $g$ is continuous on $[a, b]$ and differentiable on $(a, b)$.
    \[
    g(b) = f(b) - \frac{f(b) - f(a)}{b - a}(b - a) = f(b) - f(b) + f(a) = f(a).
    \]
    \[
    g(a) = f(a).
    \]
    By Rolle,
    there is a $c \in (a, b)$ with $g'(c) = 0$.
    \[
    g'(c) = f'(c) - \frac{f(b) - f(a)}{b - a} = 0.
    \]
\end{proof}

\begin{theorem}\label{thm:diff_inc_or_dec}
    Let $f : I \to \R$ be continuous on an interval $I$,
    differentiable in its interior.
    \begin{enumerate}[label = (\roman*)]
        \item If $f'(x) = 0$ for all $x$,
        then $f$ is constant.

        \item If $f'(x) \geq 0$
        ($\leq 0$)
        for all $x$,
        then $f$ is monotonically increasing
        (decreasing).

        \item If $f'(x) > 0$
        ($< 0$)
        for all $x$,
        then $f$ is strictly monotonically increasing
        (decreasing).
    \end{enumerate}
\end{theorem}
\begin{proof}
    Let $c < d$ be two points in $I$.
    By MVT there is an $\alpha \in (c, d)$ such that
    \[
    f(d) - f(c) = (d - c)f'(\alpha) = \begin{dcases*}
        0 & in case (i) \\
        \geq 0 & in case (ii) \\
        > 0 & in case (iii).
    \end{dcases*}
    \]
    In case (i) $f(d) = f(c)$,
    in case (ii) $f(d) \geq f(c)$,
    in case (iii) $f(d) > f(c)$.
\end{proof}

\begin{theorem}[Cauchy's Generalised Mean Value Theorem]\label{thm:genmvt}
    Let $f, g : [a, b] \to \R$ continuous and differentiable on $(a, b)$.
    Assume $g'(x) \neq 0$ for all $x \in (a, b)$.
    Then there exists $c \in (a, b)$ such that
    \[
    \frac{f'(c)}{g'(c)} = \frac{f(b) - f(a)}{g(b) - g(a)}.
    \]
\end{theorem}
\begin{proof}
    Consider
    \[
    h(x) = (g(b) - g(a))f(x) - (f(b) - f(a))g(x)
    \]
    continuous on $[a, b]$,
    differentiable on $(a, b)$.
    By Rolle there is $c \in (a, b)$ with $h'(c) = 0$
    \[
    h'(c) = (g(b) - g(a))f'(c) - (f(b) - f(a))g'(c) = 0.
    \]
\end{proof}

\begin{theorem}[L'H\^opital's Rule]\label{thm:lhopital}
    Let $f$ and $g$ be two differentiable functions on $(a, b)$.
    Assume that
    \[
    \lim_{x \to a ^ {+}}f(x) = 0
    \]
    and
    \[
    \lim_{x \to a ^ {+}}g(x) = 0
    \]
    and $g(x) \neq 0$,
    $g'(x) \neq 0$ for all $x$ on $(a, b)$.
    Then
    If
    \[
    \lim_{x \to a ^ {+}}\frac{f'(x)}{g'(x)}
    \]
    exists then also
    \[
    \lim_{x \to a ^ {+}}\frac{f(x)}{g(x)}
    \]
    exists
    and
    \[
    \lim_{x \to a ^ {+}}\frac{f(x)}{g(x)} = \lim_{x \to a ^ {+}}\frac{f'(x)}{g'(x)}.
    \]
\end{theorem}
\begin{proof}
    We can extend $f, g$ continuously to $x = a$ by setting $f(a) = g(a) = 0$.
    Take any  sequence $x_n \in (a, b)$ with $\liminfty x_n = a$.

    Need to show
    \[
    \liminfty \frac{f(x_n)}{g(x_n)} = L.
    \]
    Apply the generalised mean value theorem for $f$ and $g$ on the intervals $[a, x_n]$.
    So exists a $y_n \in (a, x_n)$ such that $\frac{f'(y_n)}{g'(y_n)} = \frac{f(x_n) - f(a)}{g(x_n) - g(a)} = \frac{f(x_n)}{g(x_n)}$.

    By squeezing $\liminfty y_n = a$
    ($a < y_n < \underbrace{x_n}_{\to a}$)
    so
    \[
    L = \liminfty\frac{f(y_n)}{g(y_n)} = \liminfty\frac{f(x_n)}{g(x_n)}.
    \]
\end{proof}

\begin{theorem}[Taylor's Theorem
(Peano remainder)]\label{thm:taylors_peano}
    $f : I \to \R$ $n$-times differentiable.
    Then there exists a function $r_n(x)$ with $\lim_{x \to c}r_n(x) = 0$ such that
    \begin{equation}\label{eq:3}
        f(x) = T_{f, c} ^ {(n)}(x) + r_n(x)(x - c) ^ n.
    \end{equation}
\end{theorem}
\begin{proof}
    Solve for $r_n(x)$ in \eqref{eq:3}.
    \[
    r_n(x) = \frac{f(x) - T_{f, c} ^{(n)}(x)}{(x - c) ^ n}.
    \]
    Need to compute $\lim_{x \to c}r_n(x)$.
    Apply L'H\^opital $n$-times get $\lim_{x \to c}\frac{f ^ {(n)}(x) - f ^ {(n)}(c)}{n!} = 0$.
\end{proof}

\begin{theorem}[Taylor's Theorem
(Lagrange remainder)]\label{thm:taylor_lagrange}
    Assume in addition that $f$ is $(n + 1)$ times differentiable.
    Then there exists a $\xi$ between $x$ and $c$ such that
    \[
    f(x) = T_{f, c} ^ {(n)}(x) + \frac{f ^ {(n + 1)}(\xi)}{(n + 1)!}(x - c) ^ {n + 1}.
    \]
\end{theorem}
\begin{proof}
    Fix $x \in I$.
    Define
    \[
    F(t) = f(x) - T_{f, t} ^ {(n)}(x)
    \]
    \[
    f(x) - \left[f(t) - f'(t)(x - t) + \dotsc + \frac{f ^ {(n)}(t)}{n!}(x - t) ^ n\right].
    \]
    So $F(c) = r_n(x)(x - c) ^ n$.
    Have $F'(t) = -\frac{f ^ {(n + 1)}(t)}{n!}(x - t) ^ n$.
    Apply Cauchy's generalised mean value theorem for $F(t)$ and $G(t) = (x - t) ^ {n + 1}$.
    Then
    \[
    \frac{r_n(x)}{x - c} = \frac{F(c)}{(x - c) ^ {n + 1}} = \frac{F(c)}{G(c)} = \frac{F(c) - \overbrace{F(x)}^{= 0}}{G(c) - \underbrace{G(x)}_{= 0}} = \frac{F'(\xi)}{G'(\xi)} = \frac{{-\frac{f ^ {(n + 1)}(\xi)}{n!}(x - \xi) ^ n}}{-(n + 1)(x - \xi) ^ n} = \frac{f ^ {(n + 1)}(\xi)}{(n + 1)!}
    \]
    with $\xi$ between $x$ and $c$.
\end{proof}

















\end{document}


\title{AnalysisTheorems}
\author{ElectrixFox}

\begin{document}
\maketitle
% In this file you should put the actual content of the blueprint.
% It will be used both by the web and the print version.
% It should *not* include the \begin{document}
%
% If you want to split the blueprint content into several files then
% the current file can be a simple sequence of \input. Otherwise It
% can start with a \section or \chapter for instance.

\chapter{realnumbers}\label{cha:realnums}

\begin{definition}[Bounded above]\label{def:boundabove}
  \lean{bound_above}
  \leanok
  Given a set $X$,
  $X$ is bounded above if there exists a $C$ such that for all $x \in X$ $x \leq C$.
\end{definition}

\begin{definition}[Supremum]\label{def:supremum}
  \lean{supremum}
  \leanok
  Let $X \subset \R$ be bounded above.
  A number $C \in \R$ is called a supremum of $X$ if $C$ is an upper bound of $X$ and whenever $B$ is an upper bound of $X$,
  then $C \leq B$.
\end{definition}

axiom completeness_axiom (X : Set ℝ) [Nonempty X] : bound_above X → ∃ C, supremum X C
\begin{axiom}[Completeness axiom]\label{ax:complax}
  \leanok
  \lean{completeness_axiom}
  \uses{def:supremum, def:boundabove}
  For any nonempty set $X$ bounded above there exists a supremum $C$ of $X$.
\end{axiom}

\begin{lemma}[Subset of bounded set is bounded]
  \label{lem:subsetboundbounded}
  \lean{subset_bound_bounded}
  \leanok
  \uses{def:boundabove}
  Given a set $X$ bounded above.
  For all subsets $Y \subset X$,
  $Y$ is bounded above.
\end{lemma}



\chapter{Sequences}\label{cha:sequences}

\begin{definition}[Limit of a sequence]\label{def:seqlimit}
    \lean{seq_is_limit} \leanok
    The limit of a sequence $x_n$ as $n \to \infty$ is
    \[
    \forall \varepsilon > 0, \exists N \in \N, \forall n \geq N, |x_n - l| < \varepsilon.
    \]
\end{definition}

\begin{theorem}[Limit of a sequence is unique]\label{thm:seqlimitunique}
    \lean{seq_uniquelim} \leanok
    If a sequence $x_n$ converges to $l$ and $m$,
    then $l = m$.
\end{theorem}
\begin{proof}
    \uses{def:seqlimit}
    To-Do.
\end{proof}

\begin{definition}[Sequence bounded above]\label{def:seqboundabove}
    \lean{seq_bound_above} \leanok
    A sequence $x_n$ is bounded above if ${x_n \mid n \in \N}$ is bounded above.
\end{definition}

\begin{definition}[Sequence bounded below]\label{def:seqboundbelow}
    \lean{seq_bound_below} \leanok
    A sequence $x_n$ is bounded below if ${x_n \mid n \in \N}$ is bounded below.
\end{definition}

\begin{definition}[Sequence bounded]\label{def:seqbound}
    \lean{seq_bounded} \leanok
    \uses{def:seqboundabove, def:seqboundbelow}
    A sequence $x_n$ is bounded if it is both bounded above and bounded below.
\end{definition}

\begin{lemma}[Convergent sequence bounded below has limit bounded below]\label{lem:seqboundbelow_limboundbelow}
    \lean{conv_seq_bound_below_imp_lim_bound_below} \leanok
    Let $x_n$ be a real convergent sequence,
    if $x_n$ is bounded below,
    then the limit of $x_n$ is bounded below.
\end{lemma}
\begin{proof}
    \uses{def:seqlimit, def:seqboundbelow}
    To-Do.
\end{proof}

\begin{lemma}[Convergent sequence bounded above has limit bounded above]\label{lem:seqboundabove_limboundabove}
    \lean{conv_seq_bound_above_imp_lim_bound_above} \leanok
    Let $x_n$ be a real convergent sequence,
    if $x_n$ is bounded above,
    then the limit of $x_n$ is bounded above.
\end{lemma}
\begin{proof}
    \uses{def:seqlimit, def:seqboundabove}
    To-Do.
\end{proof}

\begin{theorem}[Harmonic sequence converges]\label{thm:seq_harmonic_conv}
    \lean{harmonic_seq_conv} \leanok
    The sequence $x_n = \frac{1}{n}$ converges to zero.
\end{theorem}
\begin{proof}
    \uses{def:seqlimit, thm:archimedes}
    To-Do.
\end{proof}

\begin{theorem}[Convergent sequence is bounded]\label{thm:seq_conv_bound}
    \lean{conv_seq_is_bounded} \leanok
    If a sequence $x_n$ converges to $x$,
    then $x_n$ is bounded.
\end{theorem}
\begin{proof}
    \uses{def:seqlimit, def:seqbound}
    To-Do.
\end{proof}

\begin{corollary}[Convergent sequence is absolutely bounded]\label{cor:seq_conv_abs_bound}
    \lean{conv_seq_is_bounded_abs} \leanok
    If a sequence $x_n$ converges to $x$,
    then $x_n$ there exists a $C > 0$ such that for all $n \in \N$ $|x_n| \leq C$.
\end{corollary}
\begin{proof}
    \uses{thm:seq_conv_bound}
    To-Do.
\end{proof}

\begin{theorem}[Squeeze theorem (to zero)]\label{thm:squeeze_zero}
    \lean{seq_squeeze_zero} \leanok
    Let $x_n, y_n$ be real convergent sequences with $y_n \to 0$ as $n \to \infty$,
    and $\forall n \in \N, |x_n| \leq y_n$ then
    \[
    x_n \to 0 \text{ as } n \to \infty.
    \]
\end{theorem}
\begin{proof}
    \uses{def:seqlimit}
    To-Do.
\end{proof}

\begin{theorem}[Inverse sequence converges]\label{thm:seq_inv_conv}
    \lean{inv_seq_conv} \leanok
    For all $a \geq 0$ the sequence $x_n = \frac{1}{n ^ a}$ converges to zero.
\end{theorem}
\begin{proof}
    \uses{def:seqlimit, thm:seq_harmonic_conv, thm:squeeze_zero}
    To-Do.
\end{proof}

\begin{theorem}[Scalar multiplication (COLT)]\label{thm:seq_COLT_scalar_mult}
    \lean{seq_COLT_scalarmult} \leanok
    Let $x_n$ be a real convergent sequence with limit $l$,
    then for any $a \in \R$,
    \[
    a * x_n \to a * l \text{ as } n \to \infty.
    \]
\end{theorem}
\begin{proof}
    \uses{def:seqlimit}
    To-Do.
\end{proof}

\begin{theorem}[Addition of convergent sequences (COLT)]\label{thm:seq_COLT_add}
    \lean{seq_COLT_addition} \leanok
    Let $x_n, y_n$ be real convergent sequences with limits $l, m$ respectively,
    then
    \[
    x_n + y_n \to l + m \text{ as } n \to \infty.
    \]
\end{theorem}
\begin{proof}
    \uses{def:seqlimit}
    To-Do.
\end{proof}

\begin{lemma}[Linearity of convergent sequences (COLT)]\label{lem:seq_COLT_linear}
    \lean{seq_COLT_linearity} \leanok
    Let $x_n, y_n$ be real convergent sequences with limits $l, m$ respectively,
    then for all $a, b \in \R$,
    \[
    a * x_n + b * y_n \to a * l + b * m \text{ as } n \to \infty.
    \]
\end{lemma}
\begin{proof}
    \uses{thm:seq_COLT_add, thm:seq_COLT_scalarmult}
    This is a direct consequence of the previous two theorems.
\end{proof}

\begin{theorem}[Multiplication of convergent sequences (COLT)]\label{thm:seq_COLT_mult}
    \lean{seq_COLT_mult} \leanok
    Let $x_n, y_n$ be real convergent sequences with limits $l, m$ respectively,
    then
    \[
    x_n * y_n \to l * m \text{ as } n \to \infty.
    \]
\end{theorem}
\begin{proof}
    \uses{def:seqlimit, lem:seqlimboundabove_limboundabove, cor:conv_seq_is_bounded_abs}
    To-Do.
\end{proof}

\begin{theorem}[Division of convergent sequences (COLT)]\label{thm:seq_COLT_ratio}
    \lean{seq_COLT_ratio}
    Let $x_n, y_n$ be real convergent sequences with limits $l, m$ respectively,
    then if $m \neq 0$,
    and $\forall n \in \N, y_n \neq 0$,
    then
    \[
    \frac{x_n}{y_n} \to \frac{l}{m} \text{ as } n \to \infty.
    \]
\end{theorem}
\begin{proof}
    \uses{def:seqlimit, thm:seq_COLT_mult, thm:seq_COLT_scalarmult}
    To-Do.
\end{proof}

\begin{theorem}[Limit of bounded sequence is in the interval]\label{thm:seq_limit_in_interval}
    \lean{seq_limininterval} \leanok
    Let $x_n$ be a real convergent sequence with limit $l$,
    if $x_n \in (a, b)$ for all $n \in \N$,
    then $l \in (a, b)$.
\end{theorem}
\begin{proof}
    \uses{def:seqlimit, lem:seqboundabove_limboundabove, lem:seqboundbelow_limboundbelow}
    To-Do.
\end{proof}

\begin{lemma}[Inequality of two sequences implies limit inequality]\label{lem:seq_limineq}
    \lean{seq_limineq}
    Let $x_n, y_n$ be real convergent sequences with limits $x, y$ respectively,
    if $\forall n \in \N, x_n \leq y_n$,
    then $x \leq y$.
\end{lemma}
\begin{proof}
    \uses{def:seqlimit}
    To-Do.
\end{proof}

\begin{theorem}[Continuity of root]\label{thm:seq_cor}
    \lean{seq_sqrtcont}
    Let $x_n$ be a real convergent sequence,
    converging to $x$,
    with $x_n \geq 0$ for all $n \in \N$ then
    \[
    \lim_{n \to \infty} \sqrt{x_n} = \sqrt{x}.
    \]
\end{theorem}
\begin{proof}
    \uses{def:seqlimit, thm:seq_COLT_scalarmult, thm:seq_COLT_mult}
    To-Do.
\end{proof}

\begin{definition}[Sequence monotonic increasing]\label{def:seqmonoinc}
    \lean{seq_mono_inc} \leanok
    A sequence $x_n$ is monotonic increasing if
    \[
    \forall m \leq n, x m \leq x n.
    \]
\end{definition}

\begin{definition}[Sequence monotonic decreasing]\label{def:seqmonodec}
    \lean{seq_mono_dec} \leanok
    A sequence $x_n$ is monotonic decreasing if
    \[
    \forall m \leq n, x m \geq x n.
    \]
\end{definition}

\begin{theorem}[Bounded monotonically increasing sequence converges]\label{thm:seq_bound_monoinc_conv}
    If a sequence $x_n$ is monotonically increasing and bounded,
    then $x_n$ converges.
\end{theorem}
\begin{proof}
    \uses{def:seqlimit, def:seqmonoinc, def:seqbound}
    To-Do.
\end{proof}

\begin{theorem}[Bounded monotonically decreasing sequence converges]\label{thm:seq_bound_monodec_conv}
    If a sequence $x_n$ is monotonically decreasing and bounded,
    then $x_n$ converges.
\end{theorem}
\begin{proof}
    \uses{def:seqlimit, def:seqmonodec, def:seqbound}
    To-Do.
\end{proof}

\begin{theorem}[Bounded monotonic sequence converges]\label{thm:seq_mono_conv}
    \lean{seq_mono_bound_conv }
    If a sequence $x_n$ is monotonic and bounded,
    then $x_n$ converges.
\end{theorem}
\begin{proof}
    \uses{def:seqlimit, def:seqmonoinc, def:seqmonodec, def:seqbound}
    To-Do.
\end{proof}
\chapter{Subsequences}\label{cha:subsequences}

\begin{definition}[Extraction]\label{def:extraction}
    \lean{extraction} \leanok
    Let $\phi : \N \to \N$ be a strictly increasing sequence.
    $\phi_n$ is the extraction of $x_n$.
\end{definition}

\begin{definition}[Sequence decreasing]\label{def:seqdec}
    \lean{seq_dec} \leanok
    A sequence $x_n$ is decreasing if $\forall n \in \N$,
    \[
    x_{n + 1} \leq x_n.
    \]
\end{definition}

\begin{definition}[Sequence increasing]\label{def:seqinc}
    A sequence $x_n$ is increasing if $\forall n \in \N$,
    \[
    x_{n + 1} \geq x_n.
    \]
\end{definition}

\begin{lemma}[Subsequence of bounded sequence is bounded]\label{lem:seqboundsubseqbound}
    \lean{seq_bound_imp_subseq_bound} \leanok
    \uses{def:boundabove, def:boundbelow}
    Let $x_n$ be a bounded sequence.
    Then for any extraction $a$ of $x_n$,
    $x_{a_n}$ is also bounded.
\end{lemma}
\begin{proof}
    \uses{def:seqboundabove, def:seqboundbelow, def:seqbound}
    To-Do.
\end{proof}

\begin{lemma}[Subsequence greater than its index]\label{lem:subseq_ge_indx}
    \lean{subseq_ge_index} \leanok
    Let $x_n$ be a sequence.
    Let $a_n$ be an extraction of $x_n$.
    Then
    \[
    \forall j \in \N, a_j \geq j.
    \]
\end{lemma}
\begin{proof}
    \uses{def:extraction}
    To-Do.
\end{proof}

\begin{lemma}[All subsequences of a convergent sequence converge to its limit]\label{lem:subseqconvtoseqlim}
    \lean{subseq_conv_to_seq_limit} \leanok
    Let $x_n$ be a convergent sequence with limit $l$.
    Then for any extraction $a_n$ of $x_n$,
    the subsequence $x_{a_n}$ converges to $l$. 
\end{lemma}
\begin{proof}
    \uses{def:seqlimit, lem:subseq_ge_indx}
    To-Do.
\end{proof}

\begin{lemma}[Every sequence has a monotonic subsequence]\label{lem:subseqmonoincordec}
    \lean{seq_contsub_inc_or_dec}
    Let $x_n$ be a sequence.
    For any subsequence $a_n$ of $x_n$,
    $x_{a_n}$ is either increasing or decreasing.
\end{lemma}
\begin{proof}
    To-Do.
\end{proof}

\begin{theorem}[Bolzano-Weierstrass]\label{thm:bolzano-weierstrass}
    \lean{subseq_BolzanoWeierstrass} \leanok
    Every bounded sequence has a convergent subsequence.
\end{theorem}
\begin{proof}
    \uses{lem:subseqmonoincordec, def:seqlimit, thm:seq_mono_conv, lem:seqboundsubseqbound}
    To-Do.
\end{proof}
\chapter{Series}\label{cha:series}

\section{Definitions and Basic Properties}

\begin{definition}[Partial sums]\label{def:partial_sums}
  \lean{seq_partial_sums} \leanok
  The partial sums of a sequence $x_n$ is defined as
  \[
  s_n = \sum_{i = 0}^{n} x_i.
  \]
\end{definition}

\begin{definition}[Convergent series]\label{def:convergent_series}
  \lean{sum_is_limit} \leanok
  A series $\sum_{i = 0}^{\infty} x_i$ converges to $l$ if the sequence of partial sums $s_n$ converges to $l$.
\end{definition}

\begin{lemma}\label{lem:sum_tail_conv}
    \lean{sum_tail_conv} \leanok
    \uses{def:partial_sums, def:convergent_series}
    A series $\sum_{i = 0}^{\infty} x_i$ converges to $l$ if and only if the tail series $\sum_{i = N}^{\infty} x_i$ converges.
\end{lemma}
\begin{proof}
    \uses{def:convergent_series}
    To-Do.
\end{proof}

\begin{lemma}\label{lem:sumifconv_seqconvtozero}
    If $\infsumo a_k$ is convergent,
    then
    \[
    \liminfty[k] a_k = 0.
    \]
\end{lemma}
\begin{proof}
    \uses{def:convergent_series, def:seqlimit, lem:seq_COLT_linear}
    Let $S_n = \sum_{k = 0}^{n}a_k$,
    then $\liminfty S_n = s$ or some $s \in \R$.
    Write $a_n = S_n - S_{n - 1}$ for $n \geq 1$.
    By COLT
    $\lim_{n \to \infty}a_n = s - s = 0$.
\end{proof}

\begin{lemma}[Harmonic Series]\label{lem:harmsumdiv}
    The series
    \[
    \infsum\frac{1}{n}
    \]
    diverges.
\end{lemma}
\begin{proof}
    Let $S_n = \sum_{k = 1}^{n}\frac{1}{k}$ then
    \[
    S_{2 ^ n} \geq 1 + \frac{n}{2}
    \]
    so the sequence of partial sums diverges.
    Thus the series diverges.
\end{proof}

\begin{theorem}[Addition of series (COLT)]\label{thm:sum_COLT_add}
    Let $\infsumo a_k$ and $\infsumo b_k$ be convergent series with limits $a, b$ respectively.
    Then
    \[
    \infsumo (a_k + b_k) = a + b.
    \]
\end{theorem}
\begin{proof}
    \uses{def:convergent_series, def:partial_sums, thm:seq_COLT_add}
    Use COLT for sequences on the partial sums.
\end{proof}

\begin{theorem}[Scalar multiplication of series (COLT)]\label{thm:sum_COLT_scalarmult}
    Let $\infsumo a_k$ be a convergent series with limit $a$.
    Then for any $c \in \R$,
    \[
    \infsumo (c * a_k) = c * a.
    \]
\end{theorem}
\begin{proof}
    \uses{def:convergent_series, def:partial_sums, thm:seq_COLT_scalar_mult}
    Use COLT for sequences on the partial sums.
\end{proof}

\section{Convergence Criteria}

\begin{theorem}[Comparison Test]\label{thm:sum_comptest}
    Let $N \in \N$ let $(a_k)_{k \in \N_0}$, $(b_k)_{k \in \N_0}$ sequences with
    \[
    0 \leq a_k \leq b_k \text{ for } k \geq N.
    \]
    \begin{enumerate}[label = (\alph*)]
        \item If $\infsum[k = 0]b_k$ is convergent,
        then $\infsum[k = 0]a_k$ is convergent.
        \item If $\infsum[k = 0]a_k$ is divergent,
        then $\infsum[k = 0]b_k$ is divergent.
    \end{enumerate}
\end{theorem}
\begin{proof}
    \uses{def:convergent_series, def:partial_sums, thm:seq_COLT_add, lem:sum_tail_conv}
    Use $\infsum[k = 0]a_k$ is convergent if and only if $\infsum[k = N]a_k$ is convergent.

    For $n \geq N$ let $s_n = \sum_{k = N}^{n}a_k$ and $t_n = \sum_{k = N}^{n}b_k$.
    Theses are monotonically increasing.

    If $\infsum[k = 0]b_k$ is convergent so is $(t_n)_{n \geq N}$ to $t$.
    \[
    s_n \leq t_n \leq t.
    \]
    The sequence $s_n$ is monotonically increasing by $t$,
    so is convergent by Thm 2.16.

    (b)
    Since $s_n$ is monotonically increasing,
    divergence means $(s_n)$ is unbounded,
    but then $(t_n)$ is also unbounded.
\end{proof}

\begin{theorem}\label{thm:suminvseqconv}
    Let $\alpha \in \R$.
    Then
    \[
    \infsum\frac{1}{k ^ \alpha}\text{ is convergent, if and only if $\alpha > 1$}.
    \]
\end{theorem}
\begin{proof}
    \uses{thm:sum_comptest, thm:seq_inv_conv, lem:harmsumdiv}
    To-Do.
\end{proof}

\begin{definition}\label{def:alternating_series}
    A series $\infsum a_k$ is called alternating,
    if
    \[
    a_{2k} \geq 0\text{ and } a_{2k - 1} \leq 0\text{ for all $k \in \N$}
    \]
    or if 
    \[
    a_{2k} \leq 0\text{ and } a_{2k - 1} \geq 0\text{ for all $k \in \N$}.
    \]
\end{definition}

\begin{theorem}[Alternating Sign Test]\label{thm:sum_alternating_sign_test}
    Let $\dseq[k]{a}$ be a monotonically decreasing sequence of positive numbers with $\lim_{k \to \infty}a_k = 0$.
    Then the alternating series $\infsum(-1) ^ {k + 1}a_k$ is convergent.

    For the sequence of partial sums $\dseq{s}$ we have
    \[
    S_2 \leq S_4 \leq \dotsi \leq S_{2n} \leq \dotsi \leq \infsum(-1) ^ {k + 1}a_k \leq \dotsi \leq S_{2n - 1} \leq \dotsi \leq S_3 \leq S_1
    \]
    and
    \[
    \left|S_n - \infsum(-1) ^ {k + 1}a_k\right| \leq a_{n + 1}.
    \]
\end{theorem}
\begin{proof}
    \uses{def:convergent_series, def:partial_sums, thm:sum_COLT_add}
    To-Do.
\end{proof}

\section{Absolute Convergence}

\begin{definition}[Absolute convergence]\label{def:sum_abs_convergence}
    Let $\displaystyle\infsum[k = 1]a_k$ be a series,
    we call it absolutely convergent,
    if $\displaystyle\infsum[k = 1]|a_k|$ is convergent.
\end{definition}

\begin{theorem}[Absolute convergence implies convergence]\label{thm:sum_abs_convergence_implies_convergence}
    Let $\infsum[k = 1]a_k$ be absolutely convergent.
    Then the series is convergent.
\end{theorem}
\begin{proof}
    \uses{def:convergent_series, def:partial_sums, thm:sum_COLT_add}
    We have $\infsum[k = 1]|a_k|$ is convergent.
    By COLT
    \[
    \infsum[k = 1]2|a_k|\text{ is convergent}.
    \]
    We have $-|a_k| \leq a_k \leq |a_k|$.
    So $0 \leq |a_k| + a_k \leq 2|a_k|$.
    By the comparison test,
    \[
    \infsum[k = 1](a_k + |a_k|)
    \]
    is convergent.
    By COLT
    \[
    \infsum[k = 1]a_k = \infsum[k = 1]((a_k + |a_k|) - |a_k|)
    \]
    is convergent.
    Additionally,
    \[
    \infsum[k = 1]a_k \leq \infsum[k = 1]|a_k|
    \]
    we have
    \[
    \sum_{k = 1}^{n}a_k \leq \sum_{k = 1}^{n}|a_k|.
    \]
\end{proof}

\begin{theorem}[Ratio Test]\label{thm:sum_ratio_test}
    Let $\dseq[k]{a}$ be a sequence with $a_k \neq 0$ for all $k \in \N$ except finitely many.
    \begin{enumerate}[label = (\alph*)]
        \item If $\displaystyle\lim_{k \to \infty}\left|\frac{a_{k + 1}}{a_k}\right| < 1$,
        then the series $\displaystyle\infsum[k = 1]$ is absolutely convergent.
        \item If $\displaystyle\lim_{k \to \infty}\left|\frac{a_{k + 1}}{a_k}\right| > 1$,
        then the series $\displaystyle\infsum[k = 1]a_k$ is divergent.
    \end{enumerate}
\end{theorem}
\begin{proof}
    \uses{def:convergent_series, def:partial_sums, thm:seq_COLT_add}
    To-Do.
\end{proof}

\begin{theorem}[Root test]\label{thm:sum_root_test}
    For a sequence $\dseq[k]{a}$ set
    \[
    a = \limsup_{k \to \infty}\sqrt[k]{|a_k|}
    \]
    \begin{enumerate}[label = (\alph*)]
        \item If $a < 1$,
        then $\displaystyle\infsum[k = 1]a_k$ is absolutely convergent.
        \item If $a > 1$,
        then $\displaystyle\infsum[k = 1]a_k$ is divergent.
    \end{enumerate}
\end{theorem}
\begin{proof}
    \uses{def:convergent_series, def:partial_sums, thm:seq_COLT_add}
    Assume $a < 1$.
    Then for all but finitely many $k$ we have
    \[
    \sqrt[k]{|a_k|} \leq q < 1
    \]
    use $q = \frac{a + 1}{2}$.
    So for all $k \geq n_0$ we have $|a_k| \leq q ^ k$.
    By comparison test with the convergent geometric series
    \[
    \infsum[k = 1]q ^ k
    \]
    we get absolute convergence.

    Assume $a > 1$,
    then for all but finitely many $k$,
    $\sqrt[k]{|a_k|} \geq q > 1$.
    hence for all $k \geq n_1$ $|a_k| \geq q ^ k$.
    By the comparison test with the diverging geometric series
    \[
    \infsum[k = 1]q ^ k
    \]
    we get divergence.
\end{proof}

\begin{definition}[Conditionally convergent series]\label{def:cond_sum}
    Let $\infsum a_k$ be a series.
    We say this series is conditionally convergent,
    if it is convergent,
    but not absolutely convergent.
\end{definition}
\chapter{Continuity}\label{cha:continuity}

\begin{definition}[Open interval]\label{def:openinterval}
    \[
    x \in (a, b) \iff a < x < b
    \]
    \textit{$a = -\infty, b = \infty$ is fine}.
\end{definition}

\begin{definition}[Open set]\label{def:openset}
    Given $X \subseteq \R$. For all $c \in X$ there exists an open interval in $X$ containing $c$.
    
    That is,
    there exists $\delta > 0$,
    such that
    \[
    (c - \delta, c + \delta) \subseteq X.
    \]
\end{definition}

\begin{definition}[Interior point]\label{def:interiorpoint}
    An interior point $c \in X$ if there exists $(c - \delta, c + \delta) \subset X$.
\end{definition}

\begin{definition}[Closed interval]\label{def:closedinterval}
    \[
    x \in [a, b] \iff a \leq x \leq b
    \]
    ($a, b \in \R$).
\end{definition}

\begin{lemma}\label{lem:limitliesinab}
    $(x_n) \in [a, b]$ converging with $\liminfty x_n = L$.
    Then $L \in [a, b]$.
\end{lemma}
\begin{proof}
    Assume $L \notin [a, b]$;
    take $\varepsilon = \min\{|L - b|, |L - a|\}$.
    Say $L > b$.
    For all $x_n$ we have
    \begin{align*}
        |x_n - L| &= |x_n - b + b - L| \\
        &= (b - x_n) + (L - b) \\
        &> \varepsilon.
    \end{align*}
    Contradiction to $\liminfty x_n = L$.
\end{proof}

\begin{theorem}[Bolzano-Weierstrass]\label{thm:cont_bolzanoweierstrass}
    $(x_n) \in [a, b]$,
    with $a, b \in \R$,
    has a converging subsequence converging in $[a, b]$.
\end{theorem}
\begin{proof}
    $(x_n)$ are bounded which by \autoref{thm:seq_bolzanoweierstrass} has a convergent subsequence and by \autoref{lem:limitliesinab}.
\end{proof}

\begin{definition}[Compact interval]\label{def:compactinterval}
    Call $[a, b]$
    ($a, b \in \R$)
    a compact interval if the interval is bounded and closed.
\end{definition}

\section{Limits of functions}

\begin{definition}[Limit of a function]\label{def:limitfunction}
    Let $f : (a, b) \to \R$ be a function.
    Let $c \in (a, b)$ and $f$ is possibly not defined at $c$.
    We say
    \[
    \lim_{x \to c}f(x) = L
    \]
    if for all $\varepsilon > 0$ there exists a $\delta > 0$ such that
    \[
    |f(x) - L| < \varepsilon
    \]
    for all $x \neq c$ with
    \[
    |x - c| < \delta.
    \]
    We also write $f(x) \to L$ as $x \to c$.
\end{definition}

\begin{definition}[Limit from the right]\label{def:limitfromright}
    \[
    \lim_{x \to c ^ {+}}f(x) \text{ same but all } x > c.
    \]
\end{definition}


\begin{definition}[Limit from the left]\label{def:limitfromleft}
    \[
    \lim_{x \to c ^ {-}}f(x) \text{ same but all } x < c.
    \]
\end{definition}

\begin{definition}[Infinite limit]\label{def:infinlimit}
    $\liminfty[x] f(x) = L$:
    for all $\varepsilon > 0$ there exists a $k \in \R$ such that
    \[
    |f(x) - L| < \varepsilon\qquad\text{for all } x > k.
    \]
\end{definition}

\begin{proposition}[Limit of a function and sequences]\label{prop:limitfunctionandseq}
    \[
    \lim_{x \to c}f(x) = L
    \]
    \[
    \iff
    \]
    for all sequences $(x_n)$ with $\liminfty x_n = c$ have $\liminfty f(x_n) = L$.
\end{proposition}
\begin{proof}
    \uses{def:limitfunction, def:seqlimit}
    "$\implies$".
    Assume $\lim_{x \to c}f(x) = L$.
    Take $(x_n) \in (a, b)$ $(x_n \neq c)$ with $\liminfty x_n = c$.
    Take $\varepsilon > 0$.
    Need an $N$ such that
    \[
    |f(x_n) - L| < \varepsilon
    \]
    for all $n \geq N$.
    We know there exists a $\delta > 0$ such that
    \begin{equation}\label{eq:1}
        |f(x) - L| < \varepsilon
    \end{equation}
    for all $|x - c| < \delta$ $(x \neq c)$.
    Since $\liminfty x_n = c$ there exists an $N$ such that $|x_n - c| < \delta$ for all $n \geq N$.
    By \eqref{eq:1} $|f(x_n) - L| < \varepsilon$ for all $n \geq N$.

    "$\impliedby$".
    By contrapositive.
    Assume $\lim_{x \to c}f(x) \neq L$
    (or does not exist).
    Need to find a sequence $x_n$ where $\liminfty x_n = c$ but $\liminfty f(x_n) \neq L$
    (or does not exist).
    Hence there exists a $\varepsilon > 0$ such that for all $\delta > 0$ such that there exists an $x$ with $|x - c| < \delta$ but $|f(x) - L| \geq \varepsilon$.
    Take the "bad" $\varepsilon > 0$.
    Take $\delta = 1 / n$,
    get an $x = x_n$ with $|x_n - c| < \delta = \frac{1}{n}$ but $|f(x_n) - L| \geq \varepsilon$.
    \[
    \liminfty x_n = c
    \]
    but
    \[
    \liminfty f(x_n) \neq L.
    \]
    This completes the proof by the contrapositive.
\end{proof}

\begin{lemma}[Linear combination of limits]\label{lem:linearcombinationlimits}
    We have $\lim_{x \to c}f(x) = L_1$ and $\lim_{x \to c}g(x) = L_2$. Then
    \[
    \lim_{x \to c}(af(x) + bg(x)) = aL_1 + bL_2.
    \]
\end{lemma}
\begin{proof}
    Using the previous proposition and applying COLT for sequences.
\end{proof}

\begin{lemma}[Product of limits]\label{lem:productlimits}
    We have $\lim_{x \to c}f(x) = L_1$ and $\lim_{x \to c}g(x) = L_2$. Then
    \[
    \lim_{x \to c}(f(x)g(x)) = L_1L_2.
    \]
\end{lemma}
\begin{proof}
    Take $x_n \to c$. By COLT for sequences,
    \[
    \liminfty\left[f(x_n)g(x_n)\right] = \lim_{x \to c}f(x_n) \cdot \liminfty g(x_n) = L_1L_2.
    \]
\end{proof}

\begin{lemma}[Quotient of limits]\label{lem:quotientlimits}
    We have $\lim_{x \to c}f(x) = L_1$ and $\lim_{x \to c}g(x) = L_2$ with $L_2 \neq 0$. Then
    \[
    \lim_{x \to c}\left(\frac{f(x)}{g(x)}\right) = \frac{L_1}{L_2}.
    \]
\end{lemma}
\begin{proof}
    Using the previous proposition and applying COLT for sequences.
\end{proof}

\begin{proposition}[Squeezing]\label{prop:fun_squeeze}
    Assume $f(x) \leq g(x) \leq h(x)$.
    For all $x$ in a neighbourhood\footnote{Close to $c$.} of $c$ with
    \[
    \lim_{x \to c}f(x) = \lim_{x \to c}h(x) = L.
    \]
    Then
    \[
    \lim_{x \to c}g(x) = L.
    \]
\end{proposition}

\section{Continuous functions}

\begin{definition}[Continuity at a point]\label{def:continuityatpoint}
    $f : X \to \R$,
    $X = (a, b)$
    $c \in (a, b)$.
    Call $f(x)$ continuous at $x = c$ if
    \[
    \lim_{x \to c}f(x) = f(c).
    \]
    For all $\varepsilon > 0$,
    there exists $\delta > 0$ such that
    \[
    |f(x) - f(c)| < \varepsilon
    \]
    for all $x$ with
    \[
    |x - c| < \delta.
    \]
\end{definition}

\begin{proposition}[Continuity and limits]\label{prop:continuityandlimits}
    $f(x)$ is continuous at $x = c$ if and only if
    \[
    \liminfty f(x_n) = f\left(\liminfty x_n\right).
    \]
\end{proposition}
\begin{proof}
    \uses{def:continuityatpoint, def:seqlimit}
    To-do.
\end{proof}

\begin{proposition}[Continuity of composition]\label{prop:continuityofcomposition}
    Assume $f$ is continuous at $c \in X$ and $g$ is continuous at $f(c) \in Y$.
    Then $g \circ f(x)$ is continuous at $x = c$.
\end{proposition}
\begin{proof}
    Use the sequence criterion:
    take $x_n \in X$ with $\liminfty x_n = c$.
    Need $\liminfty g \circ f(x_n) = g(f(c))$.
    
    Set $y_n = f(x_n)$ since $f$ is continuous at $c$ we have that $\liminfty f(x_n) = \liminfty y_n = f(c)$,
    this sequence,
    $f(x_n)$,
    is in $Y$.
    Since $g$ is continuous at $f(c)$ we have $\liminfty g(y_n) = \liminfty g(f(c)) = \liminfty g(f(x_n))$.
\end{proof}

\section{Great Theorems}
\begin{theorem}[Intermediate Value Theorem]\label{thm:IVT}
    $f : [a, b] \to \R$ continuous.
    with $f(a) < f(b)$
    (say).
    Pick $d \in [f(a), f(b)]$;
    $f(a) \leq d \leq f(b)$.
    Then there exists a $c \in [a, b]$
    (not necessarily unique)
    such that $f(c) = d$.
\end{theorem}
\begin{proof}
    Pick $d$,
    assume $d < f(b)$
    (otherwise can pick $c = b$).
    
    Define the set
    \[
    X \coloneqq \{x \in [a, b]; f(x) \leq d\}.
    \]
    $X \neq \emptyset$,
    since $a \in X$ and bounded as a subset of $[a, b]$.
    Hence has a supremum,
    $c$.
    (By term $1$)
    exists a sequence $x_n \in X$ such that $\liminfty x_n = c$.
    $x_n \in X \subseteq [a, b]$ hence $c = \liminfty x_n \in [a, b]$.
    By continuity $\liminfty f(x_n) = f\left(\liminfty x_n\right) = f(c)$.

    Claim:
    $f(c) = d$.

    Assume not,
    i.e. $f(c) < d$\footnote{Since $f(c) = \liminfty f(x_n) \in X$ hence $\liminfty f(x_n) \leq d$}.
    Then by problem sheet $1$
    (this term)
    Q7,
    there exists a
    (small)
    neighbourhood $(c - \delta, c + \delta) \in (a, b)$ such that $f(x) < d$ for all $x \in (c - \delta, c + \delta)$.

    In particular, $f(c + \delta / 2) < d$ so $c + \delta / 2 \in X$
    but $c < c + \delta / 2$ but $c = \sup{X}$ contradiction!

    So $f(c) =  d$.
\end{proof}

\begin{corollary}\label{cor:IVT_imagecont}
    $f : I \to \R$ continuous on an interval $I$.
    Then the image $f(I)$ is also an interval.
\end{corollary}
\begin{proof}
    An interval $J$ is a set such that whenever $x < y \in J$,
    then all numbers in between are also in $J$.
    Now apply Intermediate Value Theorem.

    \textit{Use $x = f(a), y = f(b)$ and apply IVT.}
\end{proof}

\begin{theorem}\label{thm:IVT_maxmin}
    $f : [a, b] \to \R$ is continuous.
    Then $f$ takes minimum and maximum on $[a, b]$.
\end{theorem}
\begin{proof}
    Only do maximum.

    Step $1$.
    
    $f$ is bounded above,
    on $[a, b]$.

    Say it did,
    then given $n \in \N$,
    exists $x_n \in [a, b]$ such that $f(x_n) > n$ by Bolzano-Weierstrass there exists a convergent subsequence $x_{n_i}$ with limit $c \in [a, b]$,
    here we use closed interval.
    So $f(n_i) \to f(c) \in \R$ with $f(n_i) > n_i$,
    at some point $n_i > c$.

    Hence $\sup\{f(a, b)\}$ exists in $\R$,
    $M = \sup\{f(a, b)\}$.
    (By term $1$)
    there exists a sequence $y_n \in f([a, b])$ such that $\liminfty y_n = M$,
    but $y_n = f(x_n)$ by continuity $\liminfty f(x_n) = M$.

    $x_n$ might not converge but by Bolzano-Weierstrass will have a converging subsequence in $[a, b]$ so $\lim x_{n_i} = c \in [a, b]$.

    Then $f(c) = f(\lim x_{n_i}) = \lim f(x_{n_i}) = \lim y_{n_i} = M$.

    Together with the Intermediate Value theorem we get the image of a continuous function on a compact interval is again a compact interval.
\end{proof}

\begin{definition}[Continuity on a set]\label{def:continuityonset}
    Continuity on a set $X$,
    for all $c \in X$ and for all $\varepsilon > 0$,
    there exists $\delta > 0$ for all $x \in X$ with
    \[
    |x - c| < \delta \implies |f(x) - f(c)| < \varepsilon.
    \]
\end{definition}

\begin{definition}[Uniform continuity]\label{def:uniformcontinuity}
    $f : X \to \R$ is uniform continuous if for all $\varepsilon > 0$ there exists $\delta > 0$ such that for all $x, y \in X$ with
    \[
    |x - y| < \delta \implies |f(x) - f(y)| < \varepsilon.
    \]
    \textit{In other words}
    \[
    \forall \varepsilon > 0, \exists \delta > 0 \text{ s.t. } \forall x, y \in X, |x - y| < \delta \implies |f(x) - f(y)| < \varepsilon.
    \]
\end{definition}

\begin{theorem}\label{thm:contoncompactisuniform}
    $f : [a, b] \to \R$ continuous on a compact interval.
    Then $f$ is uniformly continuous.
\end{theorem}
\begin{proof}
    Assume not.
    There exists $\varepsilon > 0$ such that for all $\delta > 0$ there exists $x, y \in X$ with
    \[
    |x - y| < \delta
    \]
    but
    \[
    |f(x) - f(y)| \geq \varepsilon.
    \]
    Take such a "bad" $\varepsilon > 0$.
    So for $\delta = \delta_n = \frac{1}{n}$,
    has $x_n, y_n \in [a, b]$ with $|x_n - y_n| < \delta$ but $|f(x_n) - f(y_n)| \geq \delta$.
    By Bolzano-Weierstrass for a converging subsequence $(x_{n_i})$ of the $(x_n)$
    (since $x_n \in [a, b]$)
    say $\lim x_{n_i} = x ^ {*} \in [a, b]$.

    Claim:
    also $\lim y_{n_i} = x ^ {*}$.
    Indeed,
    \begin{align*}
        |x ^ {*} - y_{n_i}| &= |x ^ {*} - x_{n_i} + x_{n_i} - y_{n_i}| \\
        &\leq |x ^ {*} - x_{n_i}| + |x ^ {*} - y_{n_i}| \\
        &\to 0 + 0 = 0
    \end{align*}
    Squeezing gives the claim.

    Claim:
    \[
    \lim f(x_{n_i}) - f(y_{n_i}) = 0.
    \]
    Indeed
    \begin{align*}
        |f(x_{n_i}) - f(y_{n_i})| &= |f(x_{n_i}) - f(x ^ {*}) + f(x ^ {*}) - f(y_{n_i})| \\
        &\leq |f(x_{n_i}) - f(x ^ {*})| + |f(x ^ {*}) - f(y_{n_i})| \\
        &\to 0 + 0 = 0
    \end{align*}
    by continuity of $f$ and $x_{n_i}, y_{n_i} \to x ^ {*}$.

    So for $n_i$ sufficiently large
    \[
    |f(x_{n_i}) - f(y_{n_i})| < \varepsilon
    \]
    contradiction!
\end{proof}

\section{Inverse functions}
Assume $f : X \to \R$ is injective so the inverse function $f ^ {-1} : f(x) \to \R$ exists,
$f(x) = Y$.

Principle question:

if $f$ is "nice"
(e.g. continuous)
is the inverse also nice?

\begin{theorem}\label{thm:inversefunction}
    Let $f : I \to \R$ be a continuous function on an interval $I$,
    and injective
    (1-1)
    so $f(I) = J$ is also an interval and the inverse function $f ^ {-1} : J \to I$ is also continuous.
\end{theorem}
\begin{proof}
    One of the key steps:
    if $f$ is continuous and 1-1.
    Then $f$ is either strictly monotonically increasing or decreasing.
\end{proof}
\chapter{Differentiability}\label{cha:differentiability}

\begin{definition}[Derivative of a function]\label{def:differentiable}
    $f : X \to \R$
    ($X$ open).
    We say that $f$ is differentiable at a point $c \in X$ if
    \[
    \lim_{x \to c}\frac{f(x) - f(c)}{x - c}
    \]
    exists.
    If so,
    we write $f'(c)$ for the limit.
\end{definition}

\begin{lemma}[First order Taylor]\label{lem:firstordertaylor}
    $f : X \to \R$,
    $f$ is differentiable at $c$ if and only if there exists a constant $n \in \R$ and a function $r(x)$ on $X$ such that
    \begin{equation}\label{eq:2}
        f(x) = f(c) + m(x - c) + r(x)(x - c)
    \end{equation}
    with $r(x)$ is continuous at $c$ and $\lim_{x \to c} r(x) = r(c) = 0$.
    In that case $m = f'(c)$.
\end{lemma}
\begin{proof}
    \uses{def:differentiable}
    "$\implies$":
    Set $m = f'(c)$ and
    \[
    r(x) \coloneqq \begin{cases}
        \frac{f(x) - f(c) - m(x - c)}{x - c} & x \neq c, \\
        0 & x = c.
    \end{cases}
    \]
    \eqref{eq:2} holds by construction.
    Need to show
    \[
    \lim_{x \to c}r(x) = 0 = r(c),
    \]
    \begin{align*}
        \lim_{x \to c}\left(\frac{f(x) - f(c)}{x - c} - m\right) &= \lim_{x \to c}\left(\frac{f(x) - f(c)}{x - c} - f'(c)\right) = 0.
    \end{align*}

    "$\impliedby$":
    \begin{align*}
        0 &= r(c) \\
        &= \lim_{x \to c}r(x) \\
        &= \lim_{x \to c}\frac{f(x) - f(c) - m(x - c)}{x - c} \\
        &= \lim_{x \to c}\left(\frac{f(x) - f(c)}{x - c} - m\right).
    \end{align*}
    Only way this is possible if $\lim_{x \to c}\frac{f(x) - f(c)}{x - c}$ exists and is equal to $m$.
\end{proof}

\begin{proposition}[Continuity of differentiable functions]\label{prop:diffcont}
    $f : X \to \R$ as before.
    Then if $f$ is differentiable at $x = c$,
    then $f(x)$ is also continuous at $x = c$.
\end{proposition}
\begin{proof}
    Assume $f$ is differentiable at $x = c$.
    Then $f(x) - f(c) = (x - c) \frac{f(x) - f(c)}{x - c} \xrightarrow[x \to c]{} 0 \cdot f'(c) = 0$.
    So $\lim_{x \to c}f(x) = f(c)$,
    that is exactly continuity.
\end{proof}

\begin{theorem}\label{thm:sum_diff_atc}
    $f, g$ are differentiable at $x = c$.
    Then $f(x) + g(x)$ and $\alpha f(x)$ are differentiable at $x = c$.
    With
    \[
    (f + g)'(c) = f'(c) + g'(c)
    \]
    and
    \[
    (\alpha f)'(c) = \alpha f'(c).
    \]
    Also the product $f(x)g(x)$ with
    \[
    (f(x)g(x))'(c) = f'(c)g(c) + f(c)g'(c).
    \]
    Assume $f(c) \neq 0$.
    Then $\frac{1}{f(x)}$ is defined in a open neighbourhood around $x = c$ and is differentiable with
    \[
    \left(\frac{1}{f(c)}\right)' = \frac{-f'(c)}{f ^ 2(c)}
    \]
\end{theorem}
\begin{proof}
    For the product.
    Write
    \[
    f(x) = f(c) + (x - c)f_1(x)
    \]
    \[
    g(x) = g(c) + (x - c)g_1(x)
    \]
    with $f_1, g_1$ continuous at $x = c$ and $\lim_{x \to c}f_1(x) = f'(c)$ and $\lim_{x \to c}g_1(x) = g'(c)$.
    Then
    \begin{align*}
        f(x)g(x) &= (f(c) + (x - c)f_1(x))(g(c) + (x - c)g_1(x)) \\
        &= f(c)g(c) + (x - c)(f_1(x)g(c) + f(c)g_1(x) + (x - c)f_1(x)g_1(x)) \\
        \intertext{with $x = c$}
        &= f'(c)g'(c) + f(c)g'(c) + 0.
    \end{align*}
\end{proof}

\begin{theorem}[Chain rule]\label{thm:chainrule}
    $g : X \to Y \subseteq \R$,
    $f : Y \to \R$,
    $X, Y$ open,
    $g$ is differentiable at $x = c$,
    $f$ is differentiable at $y = d = g(c)$.

    Then the composition
    \[
    f \circ g(x) : X \to \R
    \]
    is differentiable at $x = c$ and
    \[
    (f \circ g)'(c) = g'(c)f'(g(c)).
    \]
\end{theorem}
\begin{proof}
    $g$ differentiable at $c$:
    \[
    g(x) = g(c) + g_1(x)(x - c)
    \]
    continuous at $c$ and $g_1(c) = \lim_{x \to c}g_1(x) = g'(c)$.

    $f$ differentiable at $g(c) = d$:
    \[
    f(y) = f(g(c)) + f_1(y)(y - g(c)).
    \]
    So
    \begin{align*}
        f \circ g(x) &= f(g(x)) \\
        &= f(g(c)) + f_1(g(x))(g(x) - g(c)) \\
        &= f(g(c)) + f_1(g(x))[g(c) + g_1(x)(x - c) - g(c) ^ 2] \\
        &= f(g(c)) + f_1(g(x))g_1(x)(x - c)
        \intertext{have $h(x) = f \circ g(x)$,
        $h_1(x) = f_1(g(x))g_1(x)$}
        &= h(c) + h_1(x)(x - c)
    \end{align*}
    since $f_1$ is continuous at $g(c)$ and $g_1$ is continuous at $c$,
    \begin{align*}
        \lim_{x \to c}h_1(x) &= f_1(g(c))g_1(c)
        \intertext{by continuity}
        &= f'(g(c))g'(c).
    \end{align*}
\end{proof}

\begin{lemma}\label{lem:e_ineq}
    \[
    x \leq e ^ x - 1 \leq \frac{x}{1 - x}\qquad(x < 1).
    \]
\end{lemma}
\begin{proof}
    To-Do.
\end{proof}

\begin{theorem}\label{thm:diff_inv_func_rule}
    $f : I \to \R$
    ($I$ an interval)
    continuous and differentiable at $x = c$,
    and $f'(c) \neq 0$.
    Assume $f$ is $1$-$1$
    (invertible).
    So
    \[
    f ^ {-1} = g : \underset{= f(I)}{Y} \to \R
    \]
    exists and is differentiable at $y = d = f(c)$ and
    \[
    (f ^ {-1})'(d) = \frac{1}{f'(f ^ {-1}(d))} = \frac{1}{f'(c)}.
    \]
\end{theorem}
\begin{proof}
    Simple case.
    Assume you knew that the inverse function $f ^ {-1}$ is differentiable.
    Then $f ^ {-1} \circ f(x) = x$ by the chain rule
    \[
    (f ^ {-1})'(f(x))f'(x) = 1
    \]
    \[
    (f ^ {-1}(f(x)))' = \frac{1}{f'(x)}
    \]
    write $x = f ^ {-1}(y)$,
    \[
    (f ^ {-1})'(y) = \frac{1}{f'(f ^ {-1}(y))}.
    \]
\end{proof}

\begin{proposition}\label{prop:diff_max_or_min}
    If $f$ is differentiable at $c$ and it has a local maximum or a local minimum at $C$,
    then $f'(c) = 0$.
\end{proposition}
\begin{proof}
    $f'(c) = \lim_{x \to c}\frac{f(x) - f(c)}{x - c}$.
    If $x > c$,
    but $x$ is near $c$,
    then $f(x) \leq f(c)$ since $c$ is a local maximum.
    In particular $\frac{f(x) - f(c)}{x - c} \leq 0$.
    Similarly,
    for $x < c$,
    $\frac{f(x) - f(c)}{x - c} \geq 0$ so $\lim_{x \to c ^ {+}}\frac{f(x) - f(c)}{x - c} \leq 0, \lim_{x \to c ^ {-}}\frac{f(x) - f(c)}{x - c} \geq 0 \implies f'(c) = 0$.
\end{proof}

\begin{theorem}[Rolle's Theorem]\label{thm:rolles}
    Let $f : [a, b] \to \R$ be continuous and differentiable on $(a, b)$,
    and suppose $f(a) = f(b)$.
    Then there exists $c \in (a, b)$ with $f'(c) = 0$.
\end{theorem}
\begin{proof}
    A continuous function on a closed interval attains a maximum and a minimum.
    So there is a $c \in [a, b]$ with $f(c) \geq f(x)$ for all $x \in [a, b]$,
    $d \in [a, b]$ with $f(d) \leq f(x)$ for all $x \in [a, b]$.
    If $c \in (a, b)$,
    we get $f'(c) = 0$ by last result.
    If $c = a$ or $c = b$.
    Then look at minimum $d$ if $f \in (a, b)$ we can use the last result again $f'(d) = 0$.
    If $d = a$ or $d = b$,
    then $f(d) = f(c)$ and the whole function is constant.
    Then $f'(x) = 0$ for all $x \in (a, b)$.
\end{proof}

\begin{theorem}[Mean Value Theorem]\label{thm:mean_value_thm}
    Let $f : [a, b] \to \R$ be continuous and differentiable on $(a, b)$.
    Then there exists a $c \in (a, b)$ such that $f'(c) = \frac{f(b) - f(a)}{b - a}$.
\end{theorem}
\begin{proof}
    $g(x) = f(x) - \frac{f(b) - f(a)}{b - a}(x - a)$.
    Then $g$ is continuous on $[a, b]$ and differentiable on $(a, b)$.
    \[
    g(b) = f(b) - \frac{f(b) - f(a)}{b - a}(b - a) = f(b) - f(b) + f(a) = f(a).
    \]
    \[
    g(a) = f(a).
    \]
    By Rolle,
    there is a $c \in (a, b)$ with $g'(c) = 0$.
    \[
    g'(c) = f'(c) - \frac{f(b) - f(a)}{b - a} = 0.
    \]
\end{proof}

\begin{theorem}\label{thm:diff_inc_or_dec}
    Let $f : I \to \R$ be continuous on an interval $I$,
    differentiable in its interior.
    \begin{enumerate}[label = (\roman*)]
        \item If $f'(x) = 0$ for all $x$,
        then $f$ is constant.

        \item If $f'(x) \geq 0$
        ($\leq 0$)
        for all $x$,
        then $f$ is monotonically increasing
        (decreasing).

        \item If $f'(x) > 0$
        ($< 0$)
        for all $x$,
        then $f$ is strictly monotonically increasing
        (decreasing).
    \end{enumerate}
\end{theorem}
\begin{proof}
    Let $c < d$ be two points in $I$.
    By MVT there is an $\alpha \in (c, d)$ such that
    \[
    f(d) - f(c) = (d - c)f'(\alpha) = \begin{dcases*}
        0 & in case (i) \\
        \geq 0 & in case (ii) \\
        > 0 & in case (iii).
    \end{dcases*}
    \]
    In case (i) $f(d) = f(c)$,
    in case (ii) $f(d) \geq f(c)$,
    in case (iii) $f(d) > f(c)$.
\end{proof}

\begin{theorem}[Cauchy's Generalised Mean Value Theorem]\label{thm:genmvt}
    Let $f, g : [a, b] \to \R$ continuous and differentiable on $(a, b)$.
    Assume $g'(x) \neq 0$ for all $x \in (a, b)$.
    Then there exists $c \in (a, b)$ such that
    \[
    \frac{f'(c)}{g'(c)} = \frac{f(b) - f(a)}{g(b) - g(a)}.
    \]
\end{theorem}
\begin{proof}
    Consider
    \[
    h(x) = (g(b) - g(a))f(x) - (f(b) - f(a))g(x)
    \]
    continuous on $[a, b]$,
    differentiable on $(a, b)$.
    By Rolle there is $c \in (a, b)$ with $h'(c) = 0$
    \[
    h'(c) = (g(b) - g(a))f'(c) - (f(b) - f(a))g'(c) = 0.
    \]
\end{proof}

\begin{theorem}[L'H\^opital's Rule]\label{thm:lhopital}
    Let $f$ and $g$ be two differentiable functions on $(a, b)$.
    Assume that
    \[
    \lim_{x \to a ^ {+}}f(x) = 0
    \]
    and
    \[
    \lim_{x \to a ^ {+}}g(x) = 0
    \]
    and $g(x) \neq 0$,
    $g'(x) \neq 0$ for all $x$ on $(a, b)$.
    Then
    If
    \[
    \lim_{x \to a ^ {+}}\frac{f'(x)}{g'(x)}
    \]
    exists then also
    \[
    \lim_{x \to a ^ {+}}\frac{f(x)}{g(x)}
    \]
    exists
    and
    \[
    \lim_{x \to a ^ {+}}\frac{f(x)}{g(x)} = \lim_{x \to a ^ {+}}\frac{f'(x)}{g'(x)}.
    \]
\end{theorem}
\begin{proof}
    We can extend $f, g$ continuously to $x = a$ by setting $f(a) = g(a) = 0$.
    Take any  sequence $x_n \in (a, b)$ with $\liminfty x_n = a$.

    Need to show
    \[
    \liminfty \frac{f(x_n)}{g(x_n)} = L.
    \]
    Apply the generalised mean value theorem for $f$ and $g$ on the intervals $[a, x_n]$.
    So exists a $y_n \in (a, x_n)$ such that $\frac{f'(y_n)}{g'(y_n)} = \frac{f(x_n) - f(a)}{g(x_n) - g(a)} = \frac{f(x_n)}{g(x_n)}$.

    By squeezing $\liminfty y_n = a$
    ($a < y_n < \underbrace{x_n}_{\to a}$)
    so
    \[
    L = \liminfty\frac{f(y_n)}{g(y_n)} = \liminfty\frac{f(x_n)}{g(x_n)}.
    \]
\end{proof}

\begin{theorem}[Taylor's Theorem
(Peano remainder)]\label{thm:taylors_peano}
    $f : I \to \R$ $n$-times differentiable.
    Then there exists a function $r_n(x)$ with $\lim_{x \to c}r_n(x) = 0$ such that
    \begin{equation}\label{eq:3}
        f(x) = T_{f, c} ^ {(n)}(x) + r_n(x)(x - c) ^ n.
    \end{equation}
\end{theorem}
\begin{proof}
    Solve for $r_n(x)$ in \eqref{eq:3}.
    \[
    r_n(x) = \frac{f(x) - T_{f, c} ^{(n)}(x)}{(x - c) ^ n}.
    \]
    Need to compute $\lim_{x \to c}r_n(x)$.
    Apply L'H\^opital $n$-times get $\lim_{x \to c}\frac{f ^ {(n)}(x) - f ^ {(n)}(c)}{n!} = 0$.
\end{proof}

\begin{theorem}[Taylor's Theorem
(Lagrange remainder)]\label{thm:taylor_lagrange}
    Assume in addition that $f$ is $(n + 1)$ times differentiable.
    Then there exists a $\xi$ between $x$ and $c$ such that
    \[
    f(x) = T_{f, c} ^ {(n)}(x) + \frac{f ^ {(n + 1)}(\xi)}{(n + 1)!}(x - c) ^ {n + 1}.
    \]
\end{theorem}
\begin{proof}
    Fix $x \in I$.
    Define
    \[
    F(t) = f(x) - T_{f, t} ^ {(n)}(x)
    \]
    \[
    f(x) - \left[f(t) - f'(t)(x - t) + \dotsc + \frac{f ^ {(n)}(t)}{n!}(x - t) ^ n\right].
    \]
    So $F(c) = r_n(x)(x - c) ^ n$.
    Have $F'(t) = -\frac{f ^ {(n + 1)}(t)}{n!}(x - t) ^ n$.
    Apply Cauchy's generalised mean value theorem for $F(t)$ and $G(t) = (x - t) ^ {n + 1}$.
    Then
    \[
    \frac{r_n(x)}{x - c} = \frac{F(c)}{(x - c) ^ {n + 1}} = \frac{F(c)}{G(c)} = \frac{F(c) - \overbrace{F(x)}^{= 0}}{G(c) - \underbrace{G(x)}_{= 0}} = \frac{F'(\xi)}{G'(\xi)} = \frac{{-\frac{f ^ {(n + 1)}(\xi)}{n!}(x - \xi) ^ n}}{-(n + 1)(x - \xi) ^ n} = \frac{f ^ {(n + 1)}(\xi)}{(n + 1)!}
    \]
    with $\xi$ between $x$ and $c$.
\end{proof}

















\end{document}
