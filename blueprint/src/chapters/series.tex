\chapter{Series}\label{cha:series}

\section{Definitions and Basic Properties}

\begin{definition}[Partial sums]\label{def:partial_sums}
  \lean{seq_partial_sums} \leanok
  The partial sums of a sequence $x_n$ is defined as
  \[
  s_n = \sum_{i = 0}^{n} x_i.
  \]
\end{definition}

\begin{definition}[Convergent series]\label{def:convergent_series}
  \lean{sum_is_limit} \leanok
  A series $\sum_{i = 0}^{\infty} x_i$ converges to $l$ if the sequence of partial sums $s_n$ converges to $l$.
\end{definition}

\begin{lemma}\label{lem:sum_tail_conv}
    \lean{sum_tail_conv} \leanok
    \uses{def:partial_sums, def:convergent_series}
    A series $\sum_{i = 0}^{\infty} x_i$ converges to $l$ if and only if the tail series $\sum_{i = N}^{\infty} x_i$ converges.
\end{lemma}
\begin{proof}
    \uses{def:convergent_series}
    To-Do.
\end{proof}

\begin{lemma}\label{lem:sumifconv_seqconvtozero}
    If $\infsumo a_k$ is convergent,
    then
    \[
    \liminfty[k] a_k = 0.
    \]
\end{lemma}
\begin{proof}
    \uses{def:convergent_series, def:seqlimit, lem:seq_COLT_linear}
    Let $S_n = \sum_{k = 0}^{n}a_k$,
    then $\liminfty S_n = s$ or some $s \in \R$.
    Write $a_n = S_n - S_{n - 1}$ for $n \geq 1$.
    By COLT
    $\lim_{n \to \infty}a_n = s - s = 0$.
\end{proof}

\begin{lemma}[Harmonic Series]\label{lem:harmsumdiv}
    The series
    \[
    \infsum\frac{1}{n}
    \]
    diverges.
\end{lemma}
\begin{proof}
    Let $S_n = \sum_{k = 1}^{n}\frac{1}{k}$ then
    \[
    S_{2 ^ n} \geq 1 + \frac{n}{2}
    \]
    so the sequence of partial sums diverges.
    Thus the series diverges.
\end{proof}

\begin{theorem}[Addition of series (COLT)]\label{thm:sum_COLT_add}
    Let $\infsumo a_k$ and $\infsumo b_k$ be convergent series with limits $a, b$ respectively.
    Then
    \[
    \infsumo (a_k + b_k) = a + b.
    \]
\end{theorem}
\begin{proof}
    \uses{def:convergent_series, def:partial_sums, thm:seq_COLT_add}
    Use COLT for sequences on the partial sums.
\end{proof}

\begin{theorem}[Scalar multiplication of series (COLT)]\label{thm:sum_COLT_scalarmult}
    Let $\infsumo a_k$ be a convergent series with limit $a$.
    Then for any $c \in \R$,
    \[
    \infsumo (c * a_k) = c * a.
    \]
\end{theorem}
\begin{proof}
    \uses{def:convergent_series, def:partial_sums, thm:seq_COLT_scalar_mult}
    Use COLT for sequences on the partial sums.
\end{proof}

\section{Convergence Criteria}

\begin{theorem}[Comparison Test]\label{thm:sum_comptest}
    Let $N \in \N$ let $(a_k)_{k \in \N_0}$, $(b_k)_{k \in \N_0}$ sequences with
    \[
    0 \leq a_k \leq b_k \text{ for } k \geq N.
    \]
    \begin{enumerate}[label = (\alph*)]
        \item If $\infsum[k = 0]b_k$ is convergent,
        then $\infsum[k = 0]a_k$ is convergent.
        \item If $\infsum[k = 0]a_k$ is divergent,
        then $\infsum[k = 0]b_k$ is divergent.
    \end{enumerate}
\end{theorem}
\begin{proof}
    \uses{def:convergent_series, def:partial_sums, thm:seq_COLT_add, lem:sum_tail_conv}
    Use $\infsum[k = 0]a_k$ is convergent if and only if $\infsum[k = N]a_k$ is convergent.

    For $n \geq N$ let $s_n = \sum_{k = N}^{n}a_k$ and $t_n = \sum_{k = N}^{n}b_k$.
    Theses are monotonically increasing.

    If $\infsum[k = 0]b_k$ is convergent so is $(t_n)_{n \geq N}$ to $t$.
    \[
    s_n \leq t_n \leq t.
    \]
    The sequence $s_n$ is monotonically increasing by $t$,
    so is convergent by Thm 2.16.

    (b)
    Since $s_n$ is monotonically increasing,
    divergence means $(s_n)$ is unbounded,
    but then $(t_n)$ is also unbounded.
\end{proof}

\begin{theorem}\label{thm:suminvseqconv}
    Let $\alpha \in \R$.
    Then
    \[
    \infsum\frac{1}{k ^ \alpha}\text{ is convergent, if and only if $\alpha > 1$}.
    \]
\end{theorem}
\begin{proof}
    \uses{thm:sum_comptest, thm:seq_inv_conv, lem:harmsumdiv}
    To-Do.
\end{proof}

\begin{definition}\label{def:alternating_series}
    A series $\infsum a_k$ is called alternating,
    if
    \[
    a_{2k} \geq 0\text{ and } a_{2k - 1} \leq 0\text{ for all $k \in \N$}
    \]
    or if 
    \[
    a_{2k} \leq 0\text{ and } a_{2k - 1} \geq 0\text{ for all $k \in \N$}.
    \]
\end{definition}

\begin{theorem}[Alternating Sign Test]\label{thm:sum_alternating_sign_test}
    Let $\dseq[k]{a}$ be a monotonically decreasing sequence of positive numbers with $\lim_{k \to \infty}a_k = 0$.
    Then the alternating series $\infsum(-1) ^ {k + 1}a_k$ is convergent.

    For the sequence of partial sums $\dseq{s}$ we have
    \[
    S_2 \leq S_4 \leq \dotsi \leq S_{2n} \leq \dotsi \leq \infsum(-1) ^ {k + 1}a_k \leq \dotsi \leq S_{2n - 1} \leq \dotsi \leq S_3 \leq S_1
    \]
    and
    \[
    \left|S_n - \infsum(-1) ^ {k + 1}a_k\right| \leq a_{n + 1}.
    \]
\end{theorem}
\begin{proof}
    \uses{def:convergent_series, def:partial_sums, thm:sum_COLT_add}
    To-Do.
\end{proof}

\section{Absolute Convergence}

\begin{definition}[Absolute convergence]\label{def:sum_abs_convergence}
    Let $\displaystyle\infsum[k = 1]a_k$ be a series,
    we call it absolutely convergent,
    if $\displaystyle\infsum[k = 1]|a_k|$ is convergent.
\end{definition}

\begin{theorem}[Absolute convergence implies convergence]\label{thm:sum_abs_convergence_implies_convergence}
    Let $\infsum[k = 1]a_k$ be absolutely convergent.
    Then the series is convergent.
\end{theorem}
\begin{proof}
    \uses{def:convergent_series, def:partial_sums, thm:sum_COLT_add}
    We have $\infsum[k = 1]|a_k|$ is convergent.
    By COLT
    \[
    \infsum[k = 1]2|a_k|\text{ is convergent}.
    \]
    We have $-|a_k| \leq a_k \leq |a_k|$.
    So $0 \leq |a_k| + a_k \leq 2|a_k|$.
    By the comparison test,
    \[
    \infsum[k = 1](a_k + |a_k|)
    \]
    is convergent.
    By COLT
    \[
    \infsum[k = 1]a_k = \infsum[k = 1]((a_k + |a_k|) - |a_k|)
    \]
    is convergent.
    Additionally,
    \[
    \infsum[k = 1]a_k \leq \infsum[k = 1]|a_k|
    \]
    we have
    \[
    \sum_{k = 1}^{n}a_k \leq \sum_{k = 1}^{n}|a_k|.
    \]
\end{proof}

\begin{theorem}[Ratio Test]\label{thm:sum_ratio_test}
    Let $\dseq[k]{a}$ be a sequence with $a_k \neq 0$ for all $k \in \N$ except finitely many.
    \begin{enumerate}[label = (\alph*)]
        \item If $\displaystyle\lim_{k \to \infty}\left|\frac{a_{k + 1}}{a_k}\right| < 1$,
        then the series $\displaystyle\infsum[k = 1]$ is absolutely convergent.
        \item If $\displaystyle\lim_{k \to \infty}\left|\frac{a_{k + 1}}{a_k}\right| > 1$,
        then the series $\displaystyle\infsum[k = 1]a_k$ is divergent.
    \end{enumerate}
\end{theorem}
\begin{proof}
    \uses{def:convergent_series, def:partial_sums, thm:seq_COLT_add}
    To-Do.
\end{proof}

\begin{theorem}[Root test]\label{thm:sum_root_test}
    For a sequence $\dseq[k]{a}$ set
    \[
    a = \limsup_{k \to \infty}\sqrt[k]{|a_k|}
    \]
    \begin{enumerate}[label = (\alph*)]
        \item If $a < 1$,
        then $\displaystyle\infsum[k = 1]a_k$ is absolutely convergent.
        \item If $a > 1$,
        then $\displaystyle\infsum[k = 1]a_k$ is divergent.
    \end{enumerate}
\end{theorem}
\begin{proof}
    \uses{def:convergent_series, def:partial_sums, thm:seq_COLT_add}
    Assume $a < 1$.
    Then for all but finitely many $k$ we have
    \[
    \sqrt[k]{|a_k|} \leq q < 1
    \]
    use $q = \frac{a + 1}{2}$.
    So for all $k \geq n_0$ we have $|a_k| \leq q ^ k$.
    By comparison test with the convergent geometric series
    \[
    \infsum[k = 1]q ^ k
    \]
    we get absolute convergence.

    Assume $a > 1$,
    then for all but finitely many $k$,
    $\sqrt[k]{|a_k|} \geq q > 1$.
    hence for all $k \geq n_1$ $|a_k| \geq q ^ k$.
    By the comparison test with the diverging geometric series
    \[
    \infsum[k = 1]q ^ k
    \]
    we get divergence.
\end{proof}

\begin{definition}[Conditionally convergent series]\label{def:cond_sum}
    Let $\infsum a_k$ be a series.
    We say this series is conditionally convergent,
    if it is convergent,
    but not absolutely convergent.
\end{definition}