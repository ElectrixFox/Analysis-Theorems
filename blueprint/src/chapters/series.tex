\chapter{Series}\label{cha:series}

\begin{definition}[Partial sums]\label{def:partial_sums}
  \lean{seq_partial_sums} \leanok
  The partial sums of a sequence $x_n$ is defined as
  \[
  s_n = \sum_{i = 0}^{n} x_i.
  \]
\end{definition}

\begin{definition}[Convergent series]\label{def:convergent_series}
  \lean{sum_is_limit} \leanok
  A series $\sum_{i = 0}^{\infty} x_i$ converges to $l$ if the sequence of partial sums $s_n$ converges to $l$.
\end{definition}

\begin{lemma}\label{lem:sum_tail_conv}
    \lean{sum_tail_conv} \leanok
    \uses{def:partial_sums, def:convergent_series}
    A series $\sum_{i = 0}^{\infty} x_i$ converges to $l$ if and only if the tail series $\sum_{i = N}^{\infty} x_i$ converges.
\end{lemma}
\begin{proof}
    \uses{def:convergent_series}
    To-Do.
\end{proof}

\begin{lemma}\label{lem:sumifconv_seqconvtozero}
    If $\infsumo a_k$ is convergent,
    then
    \[
    \liminfty[k] a_k = 0.
    \]
\end{lemma}
\begin{proof}
    \uses{def:convergent_series, def:seqlimit, lem:seq_COLT_linear}
    Let $S_n = \sum_{k = 0}^{n}a_k$,
    then $\liminfty S_n = s$ or some $s \in \R$.
    Write $a_n = S_n - S_{n - 1}$ for $n \geq 1$.
    By COLT
    $\lim_{n \to \infty}a_n = s - s = 0$.
\end{proof}

\begin{lemma}[Harmonic Series]\label{lem:harmsumdiv}
    The series
    \[
    \infsum\frac{1}{n}
    \]
    diverges.
\end{lemma}
\begin{proof}
    Let $S_n = \sum_{k = 1}^{n}\frac{1}{k}$ then
    \[
    S_{2 ^ n} \geq 1 + \frac{n}{2}
    \]
    so the sequence of partial sums diverges.
    Thus the series diverges.
\end{proof}

\begin{theorem}[Addition of series (COLT)]\label{thm:sum_COLT_add}
    Let $\infsumo a_k$ and $\infsumo b_k$ be convergent series with limits $a, b$ respectively.
    Then
    \[
    \infsumo (a_k + b_k) = a + b.
    \]
\end{theorem}
\begin{proof}
    \uses{def:convergent_series, def:partial_sums, thm:seq_COLT_add}
    Use COLT for sequences on the partial sums.
\end{proof}

\begin{theorem}[Scalar multiplication of series (COLT)]\label{thm:sum_COLT_scalarmult}
    Let $\infsumo a_k$ be a convergent series with limit $a$.
    Then for any $c \in \R$,
    \[
    \infsumo (c * a_k) = c * a.
    \]

lemma sum_add (a : ℕ → ℝ) {n : ℕ} : ∑ i ∈ Icc 0 (n + 1), a i = (∑ i ∈ Icc 0 n, a i) + a (n + 1) := by
  simp [sum_Icc_succ_top]

def sum_is_limit (x : ℕ → ℝ) (l : ℝ) : Prop :=
  seq_is_limit (seq_partial_sums x) l

def sum_is_limit' (a : ℕ → ℝ) (l : ℝ) (N : ℕ) : Prop :=
  sum_is_limit a (l - (∑ i ∈ Icc 0 (N - 1), a i))

def seq_partial_sums' (a : ℕ → ℝ) (N : ℕ := 1) : ℕ → ℝ :=
  fun n => (∑ i ∈ Icc N n, a i)

lemma sum_tail_conv (x : ℕ → ℝ) (n0 : ℕ) : (∃ l, sum_is_limit x l) ↔ (∃ m, sum_is_limit' x m n0) := by
  constructor
  .
    intro hl
    obtain ⟨l, hx⟩ := hl
    dsimp [sum_is_limit']
    set m := ∑ i ∈ Icc 0 (n0 - 1), x i
    use (l + m)
    simp
    rw [sum_is_limit] at hx
    apply hx
  .
    intro hx'
    obtain ⟨m, hm⟩ := hx'
    dsimp [sum_is_limit'] at hm
    set l := ∑ i ∈ Icc 0 (n0 - 1), x i
    use (m - l)

lemma seq_convto_general (x : ℕ → ℝ) (l : ℝ) (hx : seq_is_limit x l) (s : ℕ) : seq_is_limit (fun n => x (n + s)) l := by
  -- get all the usual stuff
  intro ε hε
  specialize hx ε hε
  obtain ⟨N, hN⟩ := hx
  use N
  intro n hn
  specialize hN (n + s) -- show it must be true for n + s ≥ N
  apply hN  -- apply this
  linarith  -- linearity shows us this is true

lemma seq_convto_general_iff (x : ℕ → ℝ) (l : ℝ) : seq_is_limit x l ↔ ∀ (s : ℕ), seq_is_limit (fun n => x (n + s)) l := by
  constructor
  . apply seq_convto_general  -- sorting the forward case
  . -- the backward case
    intro hx
    specialize hx 0
    simp at hx
    apply hx

lemma seq_convto_general_iff' (x : ℕ → ℝ) (l : ℝ) (s : ℕ) : seq_is_limit x l ↔ seq_is_limit (fun n => x (n + s)) l := by
  constructor
  . intro h
    apply seq_convto_general x l h  -- sorting the forward case
  . -- the backward case
    intro hs
    intro ε hε
    specialize hs ε hε
    obtain ⟨N, hN⟩ := hs
    use (N + s)
    simp at hN
    intro n hn
    specialize hN (n - s) (by omega)
    suffices (x (n - s + s) = x n) by
      rw [←this]
      apply hN
    suffices ((n - s + s) = n) by
      rw [this]
    omega


lemma sum_conv_if_seq_convto_zero (a : ℕ → ℝ) : (∃ l, sum_is_limit a l) → seq_is_limit a 0 := by
  dsimp [sum_is_limit]
  intro h
  obtain ⟨l, hl⟩ := h
  set s : ℕ → ℝ := seq_partial_sums a

  have h1 : ∀ k, a (k + 1) = s (k + 1) - s k := by
    intro k
    dsimp [s, seq_partial_sums]
    simp [sum_add]

  have h2 : seq_is_limit (fun n => a (n + 1)) 0 := by
    conv_lhs => ext n; apply h1
    have := seq_COLT_scalarmult s l hl (-1) -- getting the first colt
    simp at this
    conv_lhs => ext n; rw [sub_eq_add_neg]
    conv_rhs => rw [←sub_self l, sub_eq_add_neg]
    apply seq_COLT_addition _ _ l (-l) (by apply seq_convto_general s l hl) this

  rw [seq_convto_general_iff' a 0 1]  -- sorting out the limit
  apply h2

theorem sum_COLT_add (a b : ℕ → ℝ) (l m : ℝ) (ha : sum_is_limit a l) (hb : sum_is_limit b m) : sum_is_limit (fun k => a k + b k) (l + m) := by
  unfold sum_is_limit seq_partial_sums at * -- unfold the definitions
  conv_lhs => ext n; simp [sum_add_distrib] -- do some manipulations
  exact seq_COLT_addition _ _ _ _ ha hb -- apply normal COLT

theorem sum_COLT_scalarmult (a : ℕ → ℝ) (l : ℝ) (c : ℝ) (ha : sum_is_limit a l) : sum_is_limit (fun k => c * a k) (c * l) := by
  unfold sum_is_limit seq_partial_sums at * -- unfold the definitions
  conv_lhs =>
    ext n
    simp
    conv => rhs; ext x; rw [mul_comm]
    rw [←sum_mul, mul_comm]

  -- do some conversions to show what we need
  conv_lhs => ext n; rw [show ((c * ∑ i ∈ Icc 0 n, a i) = c * (∑ i ∈ Icc 0 n, a i)) by rfl]
  exact seq_COLT_scalarmult _ _ ha _ -- apply normal COLT