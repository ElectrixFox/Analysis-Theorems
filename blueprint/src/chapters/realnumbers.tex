\chapter{realnumbers}
\label{cha:realnums}

\begin{definition}[Bounded above]\label{def:boundabove}
  \lean{bound_above}
  \leanok
  Given a set $X$,
  $X$ is bounded above if there exists a $C$ such that for all $x \in X$ $x \leq C$.
\end{definition}

\begin{definition}[Supremum]\label{def:supremum}
  \lean{supremum}
  \leanok
  Let $X \subset \R$ be bounded above.
  A number $C \in \R$ is called a supremum of $X$ if $C$ is an upper bound of $X$ and whenever $B$ is an upper bound of $X$,
  then $C \leq B$.
\end{definition}

axiom completeness_axiom (X : Set ℝ) [Nonempty X] : bound_above X → ∃ C, supremum X C
\begin{axiom}[Completeness axiom]\label{ax:complax}
  \leanok
  \lean{completeness_axiom}
  \uses{def:supremum, def:boundabove}
  For any nonempty set $X$ bounded above there exists a supremum $C$ of $X$.
\end{axiom}

\begin{lemma}[Subset of bounded set is bounded]
  \label{lem:subsetboundbounded}
  \lean{subset_bound_bounded}
  \leanok
  \uses{def:boundabove}
  Given a set $X$ bounded above.
  For all subsets $Y \subset X$,
  $Y$ is bounded above.
\end{lemma}
