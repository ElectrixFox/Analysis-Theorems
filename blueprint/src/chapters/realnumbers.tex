\chapter{Real Numbers}\label{cha:realnums}

\begin{definition}[Bounded above]\label{def:boundabove}
  \lean{bound_above} \leanok
  Given a set $X$,
  $X$ is bounded above if there exists a $C$ such that for all $x \in X$ $x \leq C$.
\end{definition}

\begin{definition}[Bounded below]\label{def:boundbelow}
  \lean{bound_below} \leanok
  Given a set $X$,
  $X$ is bounded below if there exists a $C$ such that for all $x \in X$ $x \geq C$.
\end{definition}

\begin{definition}[Bounded]\label{def:bound}
  \lean{bound} \leanok
  \uses{def:boundabove, def:boundbelow}
  Given a set $X$,
  $X$ is bounded if it is both bounded above and bounded below.
\end{definition}

\begin{definition}[Maximum]\label{def:maximum}
  \lean{maximum} \leanok
  Let $X \subset \R$ be bounded above.
  A number $C \in \R$ is called a maximum of $X$ if $C$ is an upper bound of $X$ and for all $x \in X$, $x \leq C$.
  In other words,
  $C$ is a maximum if $\forall x \in X, x \leq C$.
\end{definition}

\begin{definition}[Minimum]\label{def:minimum}
  \lean{minimum} \leanok
  Let $X \subset \R$ be bounded below.
  A number $C \in \R$ is called a minimum of $X$ if $C$ is a lower bound of $X$ and for all $x \in X$, $x \geq C$.
  In other words,
  $C$ is a minimum if $\forall x \in X, x \geq C$.
\end{definition}

\begin{definition}[Supremum]\label{def:supremum}
  \lean{supremum} \leanok
  Let $X \subset \R$ be bounded above.
  A number $C \in \R$ is called a supremum of $X$ if $C$ is an upper bound of $X$ and whenever $B$ is an upper bound of $X$,
  then $C \leq B$.
\end{definition}

\begin{definition}[Infimum]\label{def:infimum}
  \lean{infimum} \leanok
  Let $X \subset \R$ be bounded below.
  A number $C \in \R$ is called an infimum of $X$ if $C$ is a lower bound of $X$ and whenever $B$ is a lower bound of $X$,
  then $C \geq B$.
\end{definition}

\begin{axiom}[Completeness axiom]\label{ax:complax}
  \lean{completeness_axiom} \leanok
  \uses{def:supremum, def:boundabove}
  For any nonempty set $X$ bounded above there exists a supremum $C$ of $X$.
\end{axiom}

\begin{lemma}[Subset of bounded set is bounded]\label{lem:subsetboundbounded}
  \lean{subset_bound_bounded} \leanok
  \uses{def:boundabove}
  Given a set $X$ bounded above.
  For all subsets $Y \subset X$,
  $Y$ is bounded above.
\end{lemma}

\begin{theorem}[Archimedes]\label{thm:archimedes}
  \lean{archimedes}
  Given $a, b \in \R$ with $b > 0$,
  there exists an $n \in \N$ such that $n * b > a$.
\end{theorem}
\begin{proof}
  \uses{ax:complax, def:boundabove, def:supremum}
\end{proof}

\begin{lemma}[Bounded above set negative is bounded below]\label{lem:setboundabve_neg_boundbelow}
  \lean{set_bound_above_neg_bound_below} \leanok
  Let $X$ be a set bounded above.
  Then $-X$ is bounded below.
\end{lemma}
\begin{proof}
  \uses{def:boundabove, def:boundbelow}
  To-Do.
\end{proof}

