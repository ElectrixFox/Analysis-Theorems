\chapter{Differentiability}\label{cha:differentiability}

\begin{definition}[Derivative of a function]\label{def:differentiable}
    $f : X \to \R$
    ($X$ open).
    We say that $f$ is differentiable at a point $c \in X$ if
    \[
    \lim_{x \to c}\frac{f(x) - f(c)}{x - c}
    \]
    exists.
    If so,
    we write $f'(c)$ for the limit.
\end{definition}

\begin{lemma}[First order Taylor]\label{lem:firstordertaylor}
    $f : X \to \R$,
    $f$ is differentiable at $c$ if and only if there exists a constant $n \in \R$ and a function $r(x)$ on $X$ such that
    \begin{equation}\label{eq:2}
        f(x) = f(c) + m(x - c) + r(x)(x - c)
    \end{equation}
    with $r(x)$ is continuous at $c$ and $\lim_{x \to c} r(x) = r(c) = 0$.
    In that case $m = f'(c)$.
\end{lemma}
\begin{proof}
    \uses{def:differentiable}
    "$\implies$":
    Set $m = f'(c)$ and
    \[
    r(x) \coloneqq \begin{cases}
        \frac{f(x) - f(c) - m(x - c)}{x - c} & x \neq c, \\
        0 & x = c.
    \end{cases}
    \]
    \eqref{eq:2} holds by construction.
    Need to show
    \[
    \lim_{x \to c}r(x) = 0 = r(c),
    \]
    \begin{align*}
        \lim_{x \to c}\left(\frac{f(x) - f(c)}{x - c} - m\right) &= \lim_{x \to c}\left(\frac{f(x) - f(c)}{x - c} - f'(c)\right) = 0.
    \end{align*}

    "$\impliedby$":
    \begin{align*}
        0 &= r(c) \\
        &= \lim_{x \to c}r(x) \\
        &= \lim_{x \to c}\frac{f(x) - f(c) - m(x - c)}{x - c} \\
        &= \lim_{x \to c}\left(\frac{f(x) - f(c)}{x - c} - m\right).
    \end{align*}
    Only way this is possible if $\lim_{x \to c}\frac{f(x) - f(c)}{x - c}$ exists and is equal to $m$.
\end{proof}

\begin{proposition}[Continuity of differentiable functions]\label{prop:diffcont}
    $f : X \to \R$ as before.
    Then if $f$ is differentiable at $x = c$,
    then $f(x)$ is also continuous at $x = c$.
\end{proposition}
\begin{proof}
    Assume $f$ is differentiable at $x = c$.
    Then $f(x) - f(c) = (x - c) \frac{f(x) - f(c)}{x - c} \xrightarrow[x \to c]{} 0 \cdot f'(c) = 0$.
    So $\lim_{x \to c}f(x) = f(c)$,
    that is exactly continuity.
\end{proof}