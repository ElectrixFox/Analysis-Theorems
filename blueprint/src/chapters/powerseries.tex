\chapter{Power Series}\label{cha:powerseries}

\begin{definition}\label{def:powerseries}
    \[
    \infsumo a_kx ^ k
    \]
    is a power series.
    $a_k \in \R$ and $x \in \R$.
\end{definition}

\begin{theorem}[Cauchy-Hadamard]\label{thm:powseries_cauchyhadamard}
    Given $\infsumo a_kx ^ k$,
    there exists $R \geq 0$
    (possibly $\infty$)
    ($R$ is the radius of convergence)
    such that
    \begin{enumerate}[label = (\roman*)]
        \item $\infsumo a_kx ^ k$ converges absolutely for $|x| < R$,

        \item $\infsumo a_kx ^ k$ diverges for $|x| > R$.
    \end{enumerate}
    In fact.
    If $c = \limsup\sqrt[k]{|a_k|}$ then $R = \frac{1}{c}$
    (if $c = 0, \infty$,
    $R = \infty, 0$).   
\end{theorem}
\begin{proof}
    We apply the root test to
    \[
    \infsumo a_kx ^ k
    \]
    \begin{align*}
        \limsup\sqrt[k]{|a_kx ^ k|} &= \limsup\sqrt[k]{|a_k|}|x| \\
        &= |x|\limsup\sqrt[k]{|a_k|} \\
        &= c|x| &= \begin{cases}
            < 1 & \text{if } |x| < 1 / c \\
            > 1 & \text{if } |x| > 1 / c
        \end{cases}
    \end{align*}
    with absolute convergence in the first case and divergence in the second.
\end{proof}

\begin{corollary}\label{cor:convradius}
    If $\infsumo a_kc ^ k$ converges for some $c \in \R$,
    then $\sum a_kx ^ k$ converges for all $x \in (-|c|, |c|)$.
\end{corollary}
\begin{proof}
    Must have $|c| \leq R \implies (-|c|, |c|) \subseteq (-R, R)$.
\end{proof}

\begin{proposition}\label{prop:powser_deriv_antideriv_same_radconv}
    $\infsumo a_kx ^ k$ with radius $R > 0$.
    Then the formal derivative
    \[
    \infsum ka_kx ^ {k - 1} = \frac{1}{x}\infsum ka_kx ^ k
    \]
    anti-derivative
    \[
    \infsumo \frac{1}{k + 1}a_kx ^ {k + 1}
    \]
    have the same radius of convergence $R$.
\end{proposition}
\begin{proof}
    \begin{align*}
        \limsup\sqrt[k]{k|a_k|} &= \lim\sqrt[k]{k}\limsup\sqrt[k]{|a_k|} \\
        &= 1 \cdot \limsup\sqrt[k]{|a_k|}.
    \end{align*}
    Anti-derivative the same.
\end{proof}

\begin{theorem}\label{thm:powser_cont}
    Power series are continuous in $(-R, R)$.
    
    In fact,
    they are "locally" Lipschitz,
    that is,
    for any $0 < r < \R$ there exists a constant $M = M_r$ such that
    \[
    |f(x) - f(y)| \leq M_r|x - y|
    \]
    for all $x \in [-r, r]$.
\end{theorem}
\begin{proof}
    \begin{align*}
        0 &\leq |f_n(x) - f_n(y)| \\
        &= \left|\sum_{k = 1}^{n}a_k(x ^ k - y ^ k)\right| \\
        &= |x - y|\cdot\left|\sum_{k = 1}^{n}a_k(x ^ {k - 1} + x ^ {k - 2}y + \dotsc + xy ^ {k - 2} + y ^ {k - 1})\right|
        \intertext{assume $|x|, |y| \leq r$}
        &\leq |x - y|\cdot\sum_{k = 1}^{n}|a_k|\cdot k \cdot r ^ {k - 1} \\
        &\xrightarrow[n \to \infty]{} |x - y|\cdot\sum_{k = 1}^{n}\underbrace{|a_k|\cdot k \cdot r ^ {k - 1}}_{= M_r}
    \end{align*}
    converges by \autoref{prop:powser_deriv_antideriv_same_radconv} which implies by squeezing
    \[
    |f(x) - f(y)| \leq |x - y|M_r.
    \]
\end{proof}

\begin{theorem}\label{thm:powerser_diff}
    The power series $f(x) = \infsumo a_kx ^ k$ is differentiable in $(-R, R)$ with term-wise derivative
    \[
    f'(x) = \infsum ka_kx ^ {k - 1}
    \]
    and $f ^ {(n)}(0) = n!a_n$.
\end{theorem}
\begin{proof}
    Next section.
    
    Assuming the term-wise derivative,
    we have
    \begin{align*}
        f'(0) &= \infsum ka_k0 ^ {k - 1} \underset{k = 0}{=} 1 \cdot a_1 \\
        f''(0) &= \infsum[k = 2]k(k - 1)a_kx ^ {k - 2} = 2(2 - 1)a_2 = 2 \cdot a_2 \\
        &\vdots \\
        f ^ {(n)}(0) &= \infsum[k = n]k(k - 1) \dotsi (k - n + 1)a_kx ^ {k - n} \underset{x = 0}{=} n!a_n.
    \end{align*}
\end{proof}

\begin{theorem}[Identity Theorem for Power Series]\label{thm:iden_pow_series}
    Assume
    \[
    \infsumo a_kx ^ k = \infsumo b_kx ^ k
    \]
    in some
    (small)
    neighbourhood of $x = 0$.
    Then $a_k = b_k$.
\end{theorem}
\begin{proof}
    Call $f(x) = \infsumo a_kx ^ k$,
    then $n!b_n = f ^ {(n)}(0) = n!a_n$ by \autoref{thm:powerser_diff}.
    So $a_n = b_n$ for all $n$.
\end{proof}

\begin{theorem}\label{thm:pow_series_def_e}
    \begin{enumerate}[label = (\roman*)]
        \item $\exp(0) = 1$.

        \item The exponential function is infinitely often differentiable on $\R$ with
        \[
        \exp'(x) = \exp(x).
        \]

        \item For all $x, y \in \R$ we have
        \begin{align*}
            \exp(x + y) &= \exp(x)\exp(y); \\
            \exp(-x) &= \frac{1}{\exp(x)}.
        \end{align*}

        \item $\exp(x) > 0$ for all $x \in \R$.

        \item $\exp(x)$ is strictly monotone increasing.

        \item
        \[
        \liminfty[x]\exp(x) = \infty\qquad\text{and}\qquad\lim_{x \to -\infty}\exp(x) = 0.
        \]
    \end{enumerate}
\end{theorem}
\begin{proof}\phantom{}
    \begin{enumerate}[label = (\roman*)]
        \item Trivial by substitution.

        \item
        \begin{align*}
            \left(\infsumo \frac{x ^ k}{k!}\right)' &= \infsum\frac{kx ^ {k - 1}}{k!} \\
            &= \infsumo\frac{(k + 1)x ^ k}{(k + 1)!} \\
            &= \infsumo\frac{x ^ k}{k!}.
        \end{align*}

        \item Two options:
        for the first property we can use the Cauchy product for infinite series.

        Set $f(t) = \exp(x + t)\cdot\exp(y - t)$.
        \[
        f(0) = \exp(x)\exp(y)
        \]
        \[
        f(y) = \exp(x + y) \cdot 1 = \exp(x + y)
        \]
        \[
        f'(t) = \exp(x + t)\exp(y - t) + \exp(x + t)(-1)\exp(y - t) = 0
        \]
        by the mean value theorem $f(t)$ is constant.

        For $\exp(-x) = \frac{1}{\exp(x)}$.
        Plug in $y = -x$.

        \item True for $x > 0$,
        for $x < 0$ use $\exp(x) = \frac{1}{\exp(-x)} > 0$.

        \item $\exp'(x) = \exp(x) > 0$ hence is strictly increasing.

        \item For $x > 0$ $\exp(x) > 1 + x$ $1 + x$ is unbounded so $\exp(x)$ is unbounded.
        Then for $x < 0$ reuse $\exp(x) = \frac{1}{\exp(-x)}$.
    \end{enumerate}
\end{proof}

\begin{definition}\label{def:pow_series_sin_cos}
    We define the sine and cosine function by
    \[
    \sin(x) \coloneqq \infsumo\frac{(-1) ^ k}{(2k + 1)!}x ^ {2k + 1};\qquad\cos(x) \coloneqq \infsumo\frac{(-1) ^ k}{(2k)!}x ^ {2k}.
    \]
\end{definition}

\begin{theorem}\label{thm:def_pow_series_sin_cos}
    We have
    \begin{enumerate}[label = (\roman*)]
        \item $\sin(0) = 0$ and $\cos(0) = 1$.

        \item The sine-function is odd;
        cosine is even.

        \item Sine and cosine are infinitely often differentiable on $\R$ with
        \[
        \sin'(x) = \cos(x)\qquad\text{and}\qquad\cos'(x) = -\sin(x).
        \]

        \item For all $x, y \in \R$ we have
        \begin{align*}
            \sin(x + y) &= \sin(x)\cos(y) + \cos(x)\sin(y) \\
            \cos(x + y) &= \cos(x)\cos(y) + \sin(x)\sin(y)
        \end{align*}

        \item For all $x \in \R$ we have
        \[
        \sin ^ 2(x) + \cos ^ 2(x) = 1.
        \]
        In particular,
        $|\sin(x)| \leq 1$;
        $|\cos(x)| \leq 1$ for all $x \in \R$.
    \end{enumerate}
\end{theorem}
\begin{proof}\phantom{}
    \begin{enumerate}[label = (\roman*)]
        \item By definition.
        
        \item By definition.

        \item Differentiating term-wise.

        \item
        \[
        f(t) \coloneqq \sin(x + t)\cos(y - t) + \cos(x + t)\sin(y - t).
        \]
        \begin{align*}
            f(0) &= \sin(x)\cos(y) + \cos(x)\sin(y) \\
            f(y) &= \sin(x + y)\cos(0) + \cos(x + y)\sin(0) = \sin(x + y) \\
            f'(t) &= \dotsi = 0.
        \end{align*}
        Hence $f(t)$ is constant so $f(0) = f(y)$,
        addition law for sine.

        \item
        \[
        \sin ^ 2(x) + \cos ^ 2(x) = 1
        \]
        true for $x = 0$.
        Differentiate
        \[
        2\sin(x)\cos(x) + 2\cos(x)(-\sin(x)) = 0
        \]
        by mean value theorem.
    \end{enumerate}
\end{proof}

\begin{theorem}\label{thm:cos_smallest_pos}
    The equation $\cos(x) = 0$ has a smallest positive solution.
\end{theorem}
\begin{proof}
    \[
    \cos(2) = \underbrace{1 - 2 + \frac{16}{24}}_{-\frac{2}{3}} + \underbrace{\infsum[k = 3]\frac{(-1) ^ k}{(2k)!}2 ^ {2k}}_{-+-\dotsc} < 0
    \]
    by alternating series test.
    \[
    \cos(0) = 1 > 0
    \]
    by IVT there exists a zero at $\frac{\pi}{2}$.
\end{proof}

\begin{definition}\label{def:pi_def}
    We denote twice the smallest positive root of cosine by $\pi$,
    that is,
    \[
    \cos\left(\frac{\pi}{2}\right) = 0.
    \]
\end{definition}

\begin{theorem}\label{thm:sin_eq_one}
    \[
    \sin\left(\frac{\pi}{2}\right) = 1.
    \]
\end{theorem}
\begin{proof}
    To-Do.
\end{proof}

\begin{proposition}\label{prop:pow_series_radconv_taylor}
    Assume $f(x) = \infsumo a_kx ^ k$ is given by a power series converging in $(-R, R)$,
    ($R > 0$).

    Let $c \in (-R, R)$.
    Then $T_{f, c}(x)$ converges with radius of convergence
    \[
    ||c| - R|
    \]
    and is equal to $f(x)$.
\end{proposition}
\begin{proof}
    Skipped.
    More naturally explained in the context of complex analysis.
\end{proof}